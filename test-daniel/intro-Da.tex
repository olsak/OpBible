Přehled

Autor: Daniel

Záměr:

\begitems
* Připravit babylonského krále \x/Nabuchodonozor/a na přijetí špatných zpráv kapitoly 4. budováním důvěry ve spolehlivost a pravdivost Danielových proroctví a všemohoucnost jeho Boha v předchozích kapitolách. 
* Ujistit Izraelity (zajatce i první navrátilce do Země), že Bůh panuje nad dějinami a že jeho prorok Daniel říkal pravdu, když mluvil o prodloužené době útisku před závěrečnou fází Božího království. 
* Připravit generace vzdálené budoucnosti na pronásledování, které je bude čekat v době Antiocha IV. Epifana.
* Připravit věřící v ještě vzdálenější budoucnosti na příchod Mesiáše v době čtvrtého království.
\enditems 


Datum: Krátce po 539 př.Kr.

Klíčové pravdy:

\begitems
* Daniel a jeho přátelé byli i v exilu Bohu věrni.

* Danielovi lze věřit, že říká pravdu, protože svou víru nikdy nezkompromitoval ani pod nátlakem svých otrokářů.

* Bůh je absolutní vládce nad veškerou historií.

* Otroctví Izraele je prodlouženo do doby, než se v nadvládě nad ním nevystřídají celkem čtyři  království (z nichž Babylón je první), protože Boží lid se neodvrátil od svých hříchů. 

* Přestože Izrael čeká v budoucnosti ještě spousta utrpení, Boží Pomazaný, Kristus, jednou přijde a přinese spásu.

\enditems

Autor

Autorství knihy \x/Daniel/ je mezi vykladači předmětem rozvleklých debat.
Mnoho badatelů datuje vznik knihy mezi 170 a 165 př.Kr., do doby vlády Antiocha IV. Epifana, dávno poté, co žil prorok Daniel (tzv. \uv{makabejské} datování, ssrv. čl. <"Kdo byl \x/Darius/ Médský?" 6>a na str. \pg). 
Toto datum je však v rozporu s knihou samotnou, která naznačuje, že Daniel je její hlavní autor (<9:2>; <10:2>) a že byla sepsána krátce po dobytí Babylóna \x/Cýr/em v roce 539 př.Kr. Kromě toho  Kristus sám explicitně spojuje knihu s prorokem \x/Daniel/em (<Mt 24:15>) 

Doba a místo vzniku

Spor o datování knihy \x/Daniel/ zahrnuje tři základní problémy:

\begitems \style n
* povahu proroctví,
* údajné historické chyby v Danielovi a 
* jazykové rysy hebrejštiny a aramejštiny v knize.
\enditems

Obecně vzato, izraelští proroci se primárně zabývali náboženskými a společenskými okolnostmi, které se týkaly jich samých a jejich vrstevníků. Když proroci předvídali budoucnost, většinou se týkala blízkých událostí v dohlednu.
Z toho důvodu jsou někteří interpreti toho názoru, že \x/Daniel/ova vize ohledně \uv{krále Severu} a \uv{krále Jihu} (<11:2-12:3>) je příliš podrobná na to, aby ji mohl sepsat \x/Daniel/, který žil nějakých 200--300 let před událostmi, zachycenými v proroctví.

Tento postoj však popírá nadpřirozený charakter proroctví, podobně jako v případě příležitostných praktik jiných proroků (např. <1Kr 13:2>; <Iz 44:28>). Přestože pasáž  \<Da 11:2-12:3> je neobvyklá, určitě není nemožné, aby \x/Daniel/ takové detaily znal; ostatně právě jemu Bůh zjevoval tajnosti jako nikomu jinému (srv. např. <2:19-23>).  

Někteří zastánci pozdního datování argumentují historickými nepřesnostmi, které knize \x/Daniel/ přičítají.
Zpochybňují \x/Belšazar/ův vztah k \x/Nabuchodonozor/ovi (viz <"pozn." 5:2>n>), stejně jako i identitu  \x/Daria/ Médského (viz <"pozn." 6:1>). 

Navíc identifikují čtyři království, předpovězená Danielem (kap. 2; 7), jako Babylón, Médeu, Persii a Řecko (včetně Seleucidů a Ptolemájů). Tato identifikace je však problematická, protože pro nezávislé Médské království v intervalu mezi královstvími Babylónským a Perským neexistuje historický důkaz.
Perský král \x/Cýr/ (550--530 př.Kr.) dobyl Médeu roku 549 př.Kr a Babylón 539 př.Kr. (viz poznámky <5:1> a <5:31>).

Advokáti raného datování knihy chápou sekvenci čtyř království jako předpověď Babylóna, Médo-Persie, Řecka a Říma. 
Tento pohled podporuje narážka na \uv{Médy a Peršany} v <5:28>, která prokazuje, že autor považoval oba národy za součásti jednoho království.

Podporovatelé pozdního data namítají, že se v textu vyskytuje několik termínů vypůjčených z řečtiny k označení hudebních nástrojů (viz <"pozn." 3:5>n), podobně jako i  pozdně hebrejské a aramejské výrazy (viz <"pozn." 2:4>n).
Žádný z těchto argumentů však není přesvědčivý.
Existuje nepřeberné množství důkazů o kontaktech mezi Řeky a národy Blízkého Východu před dobou Alexandra Velikého. Ty zcela postačují k vysvětlení existence minimálního počtu slov převzatých z řečtiny před Alexandrovým dobytím. 
Původní názvy hudebních nástrojů běžně provázejí své nositele bez odpovídajícího ekvivalentu v lokálním jazyce; srovnejme dnešní českou nepřekládanou terminologii, spojenou s hudebními nástroji:  \uv{gibsonka}, \uv{jumbo}, \uv{stratocaster}, \uv{telecaster}, \uv{Les Paul}, \uv{stage piano}, \uv{Hohnerka}, \uv{humbucker}, \uv{single-coil} apod.
Naopak: Zastánci makabejského datování mají problém, jak vysvětlit absolutní absenci výrazů, přejatých z řečtiny, {\it mimo\/} hudební terminologii. Kdyby kniha vznikla až za řecké vlády, obchodní, vojenská, a politická, administrativní apod. terminologie  by se hemžila řeckými pojmy. Nic takového však v knize není.

Aramejština a hebrejština knihy Daniel může být datována kdekoliv mezi pozdním šestým a raným druhým stoletím př.Kr. Jinými slovy, lingvistické důkazy nepřikládají příliš váhy žádnému z hledisek: ani pozdnímu, ani ranému datování.

Argument pro datum ve druhém století př.Kr. je v rozporu s biblickým tvrzením ohledně data a autorství knihy Daniel a pozdní datování neprokazuje dostatečně přesvědčivě.   Datum krátce po 539 př.Kr. (viz <1:21>) nejlépe odpovídá povaze proroctví, historickým datům i jazykové stránce textu.


 Záměr a zvláštnosti
 
 Daniel obsahuje dva různé druhy materiálu.
 V prvních šesti kapitolách je šest historických vyprávění; ve druhé polovině (kapitoly 7--12) jsou čtyři vize, téměř exkluzivně prediktivní. Mezi šesti příběhy první poloviny vyčnívá kapitola 2, protože také obsahuje předpověď. 

Zkoumání obsahu historických vyprávění ukazuje, že jsou to nezávislé celky, poskládané k sobě s určitým záměrem.
Vyprávění nenabízí ani historii Izraele pod babylónskou či perskou nadvládou, ani životopisné záznamy Daniela a jeho přátel. Má dva hlavní důrazy.

Na jednu stranu příběhy ukazují, jak Boží absolutní svrchovanost zasahuje do záležitostí všech národů 
(<2:47>; <3:17-18>; <4:28-37>; <5:18-31> <6:25-28>).
Jeruzalém byl v troskách, Boží lid v zajetí a bezbožní vládcové se zdáli triumfovat, avšak Bůh zůstává svrchovaný.
Podle své neochvějné vůle vstupuje mezi království tohoto světa, aby založil univerzální království, jemuž nikdy nebude konce.

Tyto příběhy zachycují Daniela a jeho přátele coby prominentní osoby v zemi svých otrokářů, ne však proto, že by zkompromitovali svou věrnost Bohu, ale naopak proto, že Boží požehnání je vyvýšilo.
To je ústřední motiv, protože dodává na důvěryhodnosti Danielovým proroctvím, zejména těm, která hovoří o prodlouženém utrpení Izraele. 

Vize kapitol 7--12 obsahují predikce budoucích časů, během nichž pravdivost vyprávění nabude pro Boží lid na významu.
Ačkoliv Izraelité pod nadvládou Babylóňanů i Peršanů trpěli, nepostihl je žádný rozšířený a systematický útok na jejich víru. Ten nastal až s Antiochem IV. Epifanem, panovníkem nad Seleukovským impériem mezi roky 175--164 př.Kr, který
usiloval vymýtit náboženství Židů a přinutit je přizpůsobit se řeckým náboženským praktikám.
Mnoho Židů ho poslechlo, ale jiní se vzepřeli a trpěli protivenství. 
Jedním z hlavních důvodů pro sepsání knihy Daniel je připravit Boží lid na dobu Antiocha Epifana a povzbudit k vytrvalosti ty, kteří budou žít v nadcházejících časech pronásledování.

Kniha také vzhlíží až za dobu Antiocha Epifana ke Kristově příchodu, který jednou zničí všechna lidská impéria a nastolí své věčné království spravedlnosti a pokoje.
Všechny tyto události mají Danielova proroctví na zřeteli.
Kniha sloužila jako mocné povzbuzení pro Boží lid, trpící útiskem, a dodnes je pronásledovaným věřícím podnětnou inspirací. 

Kristus v Danielovi

Danielova zaměřenost na obnovení Izraele po skončení vyhnanství obrací pozornost k Ježíšovi docela přímočaře.
Podobně jako i někteří jiní proroci Daniel předpovídal Božímu lidu slavnou budoucnost, jejíž naplnění Nový Zákon 
spojuje s prvním a druhým příchodem Kristovým, stejně jako s celými dějinami církve.

Detaily naplnění Danielových vizí sice obklopuje řada kontroverzí, avšak základní struktura Danielových vizí nenechává nikoho na pochybách, že naplněním prorokových nadějí je Kristus.
Nejzřetelněji je to vidět na způsobu, jakým se Ježíš označuje za \uv{Syna člověka} (např. <Mt 9:6>; <Mt 10:23>; <Mt 12:8>).
Daniel používal tento pojem ve významu Bohem vyvýšeného davidovského krále, reprezentujícího Boha na zemi.
Ježíš, Mesiáš, je ultimátním davidovským Králem; jenom on naplňuje predikce o Synu člověka v Danielových vizích (viz <"poznámky" 7:13>n a <7:14>n; viz teologický článek 
<"Království Boží"  Mt 4>a). 

Kromě toho se Daniel v 9. kapitole dozvěděl, že Jeremjášova predikce 70 let vyhnanství bude prodloužena 
na \uv{sedmdesát týdnů} let (<9:24>), tedy asi 490 let.
Tato předpověď dochází počátku svého naplnění v Kristově prvním příchodu. Prodleva koresponduje se sérií čtyř cizích impérií, která budou Boží lid utlačovat (<2:1-49>) a se skálou, která se stala \uv{horou velikou a naplnila celou zemi} (<2:35>) a kterou Daniel označuje jako \uv{království, jež nebude zničeno} (<2:44>). 
To je království Kristovo, které bylo inaugurováno jeho prvním příchodem, dodnes pokračuje a roste, a svého dovršení dosáhne při Kristově slavném návratu (viz teologické články <@"Království Boží" u Mt 4> a <@"Plán věků" u Žd 7>.)

Daniel předvídal i jiné, ještě konkrétnější události, které v Novém Zákoně znovu vstoupily do popředí.
Např. Ježíš  se odvolává na Danielovu predikci o \uv{otřesné ohavnosti} (viz <@pozn. 9:17>n; <11:31>n; <12:11>n),
která původně ukazovala na zneuctění chrámu řeckým Antiochem IV. Epifanem (viz Úvod: Záměr a zvláštnosti) coby předobraz zničení chrámu  římským generálem Titem v roce 70 po Kr. (viz <@"poznámky k" Mt 24:15>n a <Mk 13:14>n).

Většina křesťanů spojuje tuto typologii s Antikristem, jehož duch již ve světě působí (viz <@pozn. 1Jn 2:18>n) a zjeví se v plnosti, zřejmě jako konkrétní osoba, v blízkosti Kristova návratu (viz <@pozn. 2Te 2:3>n).


\endinput

%Konec intro-Da.tex



Překlad z DeepL.com:
Autor:  Daniel

Záměr: Ujistit vyhnance a první navrátilce do Země, že Bůh řídí dějiny a že
jeho prorok Daniel mluvil pravdu o prodloužené době útrap před závěrečnou fází Božího království.

Datum: Krátce po roce 539 př.Kr.

Klíčové pravdy:

- Daniel a jeho přátelé byli věrní Bohu během
v době svého vyhnanství.

- Danielovi se dalo věřit, že říká pravdu, protože
nikdy nedělal kompromisy se svými vězniteli.

- Bůh má absolutní kontrolu nad celými dějinami.

- Vyhnanství Izraele se protáhlo na vládu čtyř království.
nad Božím lidem pro jejich  hřích, který ani ve vyhnanství neopustili.

 V budoucnu měly přijít pro Izrael zkoušky, ale Pomazaný (Mesiáš), Kristus, přijde a přinese spasení.

Autor

Autorství knihy Daniel bylo v minulosti mezi vykladači předmětem značných diskusí. Mnoho
učenců datuje knihu mezi roky 170 a 165 př.Kr., tedy do doby života Antiocha IV. Epifana, dlouho
po době proroka Daniela. Nicméně toto pozdní je v rozporu se samotnou knihou, která naznačuje, že
Daniel byl jejím hlavním autorem (<9:2>; <10:2>) a že  byla napsána krátce po dobytí Babylóna 
\x/Cýr/em v roce 539 př. n. l. Kromě toho sám Kristus výslovně spojil tuto knihu s prorokem Danielem (<Mt 24:15>).

Doba a místo sepsání

Spory o datování knihy Daniel zahrnují tři základní otázky: (1) povahu proroctví,
(2) údajné historické omyly v Danielovi a (3) jazykové rysy hebrejštiny a aramejštiny v knize.

Obecně lze říci, že izraelští proroci se zabývali především náboženskými a společenskými okolnostmi
svých vlastních a svých současníků. Když proroci předpovídali budoucí události, nejčastěji se to týkalo událostí v blízkém časovém horizontu.
Z tohoto důvodu se někteří vykladači domnívali, že Danielovo vidění týkající se \uv{severního krále} a \uv{jižního krále} (<11:2-12:3)> je příliš detailní na to,
aby mohlo pocházet  od Daniela, který žil 200--300 let před událostmi, zachycenými v  jeho proroctví.

Tento názor však nebere v úvahu nadpřirozený charakter proroctví a příležitostnou
praxi jiných proroků (např. <1Kr 13:2>; <Iz 44:28>; <45:1>).
Ačkoli Daniel <11:2-12:3> je bezesporu neobvyklý,  jistě není nemožné, aby Daniel tyto podrobnosti znal.

Někteří zastánci pozdního datování také podpírají svůj názor poukazem a údajné 
historické chyby. Vznášeli pochybnosti ohledně \x/Belšasar/ova
vztahu k \x/Nabuchodonozor/ovi (viz <"pozn." 5:2>n) a
identity \x/Darei/a Médského (viz <"pozn."  6:1>n).

Kromě toho identifikovali čtyři království.
předpovězená Danielem (kap. 2 a 7) jako babylónské,
Médské, Perské a Řecké (včetně Seleukovců a Ptolemaiovců).
Tato identifikace je problematická, protože neexistují žádné důkazy pro
nezávislé médské království v období mezi Babylonským a Perským královstvím.
Kýros, perský král (550--530 př.Kr.), si v roce 549 př.Kr. podrobil Médy a v roce 539 př.Kr. 
Babylóňany (viz <"pozn.  <5:1>n, <31>n).

Zastánci raného datování knihy chápou posloupnost čtyř království jako
předpovídající babylonské království, Médsko-Perské království, řecké království a římské království.
Tento pohled podporuje  zmínka o \uv{Médech a Peršanech} v <5:28>, která ukazuje, že autor je považoval za jedno království.

Protagonisté  pozdního data  tvrdí, že několik  výrazů pro hudební nástroje jsou vypůjčeny z řečtiny (viz
<"pozn."  3:5>n) a že se v textu vyskytují také pozdější hebrejské a aramejské výrazy (viz <"pozn."  2:4>). Ani jeden z těchto
argumenty nezní přesvědčivě.
Existuje mnoho důkazů o kontaktů mezi Řeky a národy z oblasti Blízkého východu před dobou Alexandra Velikého.
Tyto kontakty jsou dostatečným vysvětlením pro používání řečtiny před dobou Alexandrovým dobyvačným tažením.
Aramejštinu a hebrejštinu v Danielovi lze datovat kdykoli mezi koncem šestého a začátkem druhého století př.Kr.
Jinými slovy, jazykové důkazy nepřidávají na váze ani ranému, ani pozdnímu datu.

Argumenty pro datum vzniku ve druhém století př.Kr. jsou v rozporu s biblickými výroky týkajícími se data
a autorství knihy Daniel, pozdní datum nepodporují dostatečně přesvědčivě.
Vznik krátce po roce 539 př. Kr. (viz <1:21>) nejlépe odpovídá povaze proroctví, historickým údajům a jazyku textu.

Záměr a zvláštnosti

Daniel obsahuje dva různé typy materiálu. Šest historických vyprávění se objevuje v kapitolách 1-6 a čtyři
vize v kapitolách 7-12.
Vize jsou téměř výhradně prediktivní (předpovídají budoucnost).
Mezi šesti vyprávěními je kapitola 2 odlišná, protože obsahuje také predikci.

Úvaha o obsahu historických vyprávění ukazuje, že se jedná o samostatné narativní celky,
které byly sloučeny do jedné knihy za určitým účelem.
Tato vyprávění neposkytují ani dějiny Izraele pod Babylonem nebo Persií, ani životopis Daniela nebo jeho přátel.
Přehled odhaluje dva hlavní zájmy.
Na jedné straně příběhy zdůrazňují, jak absolutní Boží svrchovanost působí v záležitostech
všech národů (<2:47>; <3:17-18>; <4:28-37>; <5:18-31>; <6:25-28>).
Jeruzalém byl zničen, chrám ležel v v troskách, Boží lid byl ve vyhnanství a zlí vládci zdánlivě triumfovali -- ale Bůh zůstal svrchovaný.
Podle své svrchované libosti zasahoval do království tohoto světa, aby nastolil své
univerzální království, které bude trvat navěky.

Tato vyprávění líčí Daniela a jeho přátele jako významné ve vlastech svých věznitelů, ne však
proto, že by se zpronevěřili své věrnosti Bohu, ale protože byli vyvýšeni Božím požehnáním.
Tento motiv je ústřední, protože dodává důvěryhodnosti Danielovým proroctvím, zejména těm, která se týkají
prodlouženého utrpení Izraele.

 Vize (kap. 7--12) obsahují předpovědi budoucích časů, během nichž se pravdy z vyprávění ukážou jako zvláště důležité pro Boží lid.
 Přestože byli Židé v době  podřízenosti babylónským a perským vládcům pronásledováni, 
nedocházelo k žádnému rozsáhlému a systematickému pokusu o vyhlazení jejich víry.
K tomu došlo až v době Antiocha IV. Epifana, vládce Seleukovské říše v letech 175--164 př.Kr.
Ten se pokusil vymýtit židovské náboženství a přimět Židy, aby  přijali řecké náboženské zvyklosti.
Mnoho Židů to udělalo, ale jiní odmítli a byli tvrdě pronásledováni.
Jedním z hlavních důvodů pro sepsání knihy Daniel bylo připravit Boží lid na dobu Antiocha Epifana a dodat povzbuzení těm, kteří  budou žít
v nadcházejícím období pronásledování.

Kniha se také dohlédne i za dobu Antiocha Epifana až k příchodu Krista, který zničí všechna lidská království a
a nastolí své věčné království spravedlnosti a míru.
Všechny tyto události jsou  v Danielových proroctvích obsaženy.
Tato kniha je velkým povzbuzením Božímu lidu v dobách pronásledování a nadále inspiruje ty, kteří trpí pronásledováním i dnes.


Kristus v Danielovi

Danielovo zaměření na obnovu Izraele po vyhnanství obrací pozornost  přímo k Ježíši.
Podobně jako ostatní sz proroci, i Daniel předpovídal Božímu lidu slavnou budoucnost, kterou Nový zákon vysvětluje jako
naplněnou v prvním a druhém příchodu Krista, a také v celých církevních dějinách.

Kolem mnoha detailů naplnění Danielových proroctví panují spory, ale základní
struktura Danielovy vize budoucnosti nás bez pochyby ujišťuje, že Kristus naplňuje prorokovy naděje.
To je nejzřetelněji vidět ve způsobu, jakým se Ježíš identifikuje
coby  \uv{Syn člověka} (např. <Mt 9:6>; <10:23>; <12:8>).
Ve významu v jakém Daniel tento termín používal, je \uv{syn člověka} velký Davidovský král, vyvýšený Bohem a reprezentující Boha na zemi.
Ježíš, Mesiáš, byl ultimátní Davidovský král; on jediný naplňuje proroctví o Synu člověka v Danielově vizi (viz <"pozn."  <7:13>n a <14>n;
viz teologický článek <"Boží království" Mt 4>a na str. \pg).

Kromě toho Daniel v 9. kapitole poznal, že Jeremiášova předpověď o sedmdesátiletém  vyhnanství Izraele v Babylonu
bude prodloužena na \uv{sedmdesát 'sedmic'} let (<9:24>), tedy asi na 490 let.
Tato předpověď nachází počátek svého naplnění v prvním příchodu Krista.

Prodloužení vyhnanství odpovídá sérii čtyř cizích impérií, které utlačovaly Boží lid (<2:1-49>)
a zjevení \uv{skály, která ... se stala obrovskou horou a naplnila celou zemi} (<2:35>) a
kterou Daniel později nazval \uv{královstvím, které nikdy nebude zničeno} (<2:44>).
Tímto velkým královstvím není nic jiného než církev, království Kristovo, které začalo při jeho prvním
příchodu, pokračuje i dnes a dosáhne svého dovršení při Kristově slavném návratu (viz teologické články
<"Boží království"  Mt 4>a a <"Plán věků" Žd 7>a).

Další, konkrétnější události předpovězené Danielem vystupují do popředí také v Novém zákoně.
Například sám Ježíš se odvolával na Danielovo proroctví o \uv{ohavnosti, která způsobí zpustošení}
(viz <"poznámky" 9:27>n; <11:31>n; <12:11>n), které se původně týkalo znesvěcení chrámu řeckým Antiochem IV. Epifanem
(viz Úvod: Záměr a zvláštnosti) jako předchůdce znesvěcení, které přinesl římský generálem Titus v roce  70 po Kr.
(viz <"poznámky"  Mt 24:15>n a <Mk 13:14>n).
Tak či onak se většina křesťanských vykladačů tuto typologii úzce spojuje s Antikristem, jehož duch již působí ve světě
(viz <"pozn."  1Jn 2:18>n) a dojde plného odhalení, možná jako skutečná osoba, někdy v blízkosti návratu Krista
(viz <"pozn." 2Te 2:3>n).


\endinput
Chtělo by to možná něco jako \ww, ale aby bylo možné jich psát několik do stejného odstavce, byť s jinými frázemi. Nebudou se vyhledávat v textu, ale musejí přepínat mezi verzemi.
Viz např. předposlení odstavec výše, začínající slovem Prodloužení: Slovní spojení v \uv{uvozovkách} by chtěla variovat podle překladů. 


Pak za takovýmto Úvodem bude ještě muset následovat Osnova, tu zatím nemám, vydá možná na půl stránky.

