\CommentedBook {Da}
\Note 1:1-6.28 {}={Vyprávění}  První část knihy vyzdvihuje jednak Boží absolutní svrchovanost nad královstvími tohoto světa, jedna upřímnou odevzdanou věrnost Bohu, kterou Daniel a jeho přátelé projevovali.
Daniel chtěl svým čtenářům vštípit přesvědčení, že přestože Boží lid někdy trpívá pronásledování, králové a království povstávají a hroutí se podle Božích záměrů. Daniel také učí, že Bůh se hojně odmění těm, kdo jemu, Danielovi, věnují pozornost coby Božímu mluvčímu. Tento materiál je rozdělen do šesti navzájem nezávislých vyprávění, každé v jedné kapitole a každé obsahuhjící nějaký zázrak:
Uchování rituální čistoty Danielem a jeho přáteli (<1:1-21>); \x/Nabuchodonozor/ův sen (<2:1-49>); záchrana z ohnivé pece (<3:1-30>); \x/Nabuchodonozor/ův druhý sen (<4:1-37>); soud nad \x/Belšasar/em (<5:1-31>) a záchrana Daniela ze lví jámy (<6:1-28>).

\Note 1:1-21 {}={Uchování rituální čistoty}  Prorok uvozuje kontext své knihy vyprávěním osobní historie (své i svých přátel) zajetí, vzdělání, věrnosti Bohu a služby králi \x/Nabuchodonozor/ovi.

\Note 1:1 {\x/Joakim/}  Tj. 605 př.Kr.; téhož roku, kdy \x/Nabuchodonozor/ porazil ;   asyrsko-egyptskou koalici v bitvě u Karchemiše (<Jr 46:2>) a zahájil tak mezinárodní vzestup Babylónu k moci. Po bitvě u Karchemiše \x/Nabuchodonozor/ zaútočil na \x/Joakim/a (<2Kr 24:1-2>; <2Pa 36:5-7>) a zajal Daniela a jeho přátele. 
To byla první ze tří \x/Nabuchodonozor/ových  invazí do Judska. Druhá nastala roku 597 př.Kr. (<2Kr 24:10-14>) a třetí 587 př.Kr. (<2Kr 25:1-24>). Zdánlivou diskrepanci mezi <Da 1:1> a <Jr 25:1> a <Jr 46:2> (kde \x/Jeremiáš/ umisťuje \x/Nabuchodonozor/ův útok na \x/Joakim/a do \x/Joakim/ova čtvrtého roku místo třetího) lze objasnit rozdílem mezi babylónským a židovským systémem chronologie. V babylónském systému, který používá Daniel, byl první rok vlády panovníka považován za \uv{korunovační rok} a vláda samotná se počítala až od prvního dne měsíce Nisan následujícího roku.

\Note 1:1 {Nabuchodonozor král Babylonský}  \x/Nabuchodonozor/ přivedl Babylóňany k vítězství u Karchemiše v roce 605 př.Kr. coby korunní princ a velitel armády. Krátce po tomto vítězství se ujal babylónského trůnu po smrti svého otce Nabopolasara (626--605 př.Kr.). \x/Nabuchodonozo/rova vláda (605--562 př.Kr.) tvoří většinu historického pozadí  biblických knih \x/Jeremiáš/, \x/Ezechiel/ a \x/Daniel/.     

\Note 1:2 {vydal Pán}={Hospodin vydal} Porážka Izraele Babylónem není vysvětlitelná jen pouhou vojenskou a politickou analýzou oné doby. Bůh vždy jednal svrchovaně v záležitostech národů. Babylóňany použil jako nástroj potrestání svého vlastního lidu za porušení smluvních závazků (<2Kr 17:15>, <2Kr 17:18-20>; <2Kr 21:12-15>; <2Kr 24:3-4>).

\Note 1:2 {nádobí} Odkaz na nádobí z vypleněného chrámu, nikoliv na deportaci zajatců.

\Note 1:2 {do domu boha svého}  Hlavní božstvo babylónského panteonu byl Marduk  ( srv. <Jr 50:2>).

\Note 1:4 {liternímu umění a jazyku} Babylónská literatura byla psána klínovým písmem primárně na hliněných tabulkách. Těchto dochovaných tabulek existují tisíce. Studium této literatury seznámilo Daniela a jeho přátele s polyteistickým světonázorem Babylóňanů, plného kouzlení, čarodějnictví a astrologie.

\Note 1:5 {z stolu královského} Později se \x/Joakim/ovi dostalo stejného zaopatření (<2Kr 25:27-30>).

\Note 1:6 {\x/Daniel/, \x/Chananiáš/, \x/Mizael/ a \x/Azariáš/} Charakteristická  hebrejská jména. Dvě z nich obsahují prvek EL, znamenající \uv{Bůh}, a dvě JAH, což je zkratka osobního Božího jména, které překládáme jako \uv{Hospodin}.  \x/Daniel/ znamená \uv{Můj soudce je Bůh}, \x/Chananiáš/ \uv{Hospodin je milostivý}, \x/Mizael/ \uv{Kdo je jako Bůh?} a \x/Azariáš/ \uv{Hospodin mi pomohl}.

\Note 1:7  {Baltazar}={Baltazar ... Sidrach ... Mesak ... Abednego}
     Přesný význam těchto jmen je předmětem diskusí. Převažují tyto názory: 
     \x/Baltazar/: {\it \x/Bel/} [jiné jméno pro Marduka, hlavního boha babylónského panteonu]
     {\em chraň jeho život} nebo {\em Paní, ochraňuj krále}; 
     \x/Sidrach/: {\em Velice se bojím (Boha)} nebo {\em Přikázání Aku} [sumérský měsíční bůh];
     \x/Mesak/:   {\em Jsem bezvýznamný} nebo {\em Kdo je to, co Aku?};
     \x/Abednego/: {\em Služebník zářícího.}
     
\Note 1:8 {nepoškvrňoval}={neposkvrní} Důvod, pro který byl Daniel přesvědčen, že by ho králův pokrm poskvrnil, není uveden. Pravděpodobně jídlo znamenalo porušení dietních předpisů Mojžíšova zákona  (<Lv 11:1-47>), zakazujících konzumaci vepřového nebo masa nezbaveného krve (<Lv 17:10-14>). Také mohlo zahrnovat pokrmy, obětované babylónský modlám. 

\Note 1:9 {milost a lásku u správce} Danielův osud v mnohém připomíná Josefův příběh (<Gn 39-41>).

\Note 1:12 {deset} Často se symbolickým významem dokonalosti nebo plného počtu. \dopsat % Abrahamova přímluva za sodomu, poznámka nebo článek

\Note 1:14 {uposlechl} Daniel neslíbil, že se v případě zchátralejšího zevnějšku přizpůsobí a poslechnou nařízení v rozporu s Božím Zákonem. Je možné (a ve světle dalších kapitol i docela pravděpodobné), že už tehdy byli rozhodnuti neposlechnout bezbožného vladaře a raději zemřít, než zkompromitovat víru.

\Note 1:15 {tváře jejich byly krásnější} Bůh Danielovi a jeho přátelům požehnal pro jejich věrnost Božímu Slovu (<Dt 8:3>; <Mt 4:4>). Nepřál jim smrt, která by je nejspíše čekala, kdyby byl správce s výsledkem nespokojen, a oni by přesto na svém odmítání rituálně nečisté stravy trvali (srv. <"pozn." 14>). 

\Note 1:17 {moudrost} Danielova moudrost se stala příslovečnou ještě za jeho života; Ezechiel říká králi Týru ironicky, že je moudřejší nad Daniela (<Ez 28:3>). 

\Note 1:17 {vidění a snům}={vidění a sny} Daniel převyšoval i své přátele schopností interpretovat sny, pro kterou byl vyvýšen nade všechny ostatní, podobně jako kdysi Josef u faraonova dvora (<Gn 40:8>; <Ge 41:16>). 

\Note 1:20 {desetkrát} viz (<"pozn." 12>n).

\Note 1:20 {mudrce a hvězdáře} viz (<"pozn." Gn 41:8>n).

\Note 1:21 {léta prvního Cýra krále} Tj. \x/Cýrovy/ vlády nad Babylónem, tedy 539 př.Kr. Daniel v roce 537 př.Kr. dosud žil (<10:1>); dožil se návratu Judejců ze zajetí do Země.

\Note 4:16 {srdce zvířecí} % pozor, v CEP to je 4:13!!
     \x/Nebukadnesar/  byl postižen mentální poruchou zvanou lykantropie 
     (z řeckého {\em lukos \ {\x/lukos/} --- vlk} a {\em anthropos {\x/anthropos/} --- člověk}), 
     při níž  se člověk chová jako vlk nebo i jiné zvíře. Viz též <P 4:30>. 
     \dopsat

\Note 4:16 {sedm let} 
     Sedm období neurčené délky (srv. vv. <20> a <22>). Většina interpretů se shoduje na závěru, že 
     \x/léto/ znamená jeden rok. Verš <30> naznačuje, že doba byla delší než den, týden či měsíc.

\Note 4:33 {bylinu jako vůl jedl}
     Vzhledem k tomu, že  se \x/Nebúkadnesar/ projevoval rysy charakteristickými pro býložravce,
     je jeho mentální porucha někdy nazývána {\it boantropií.\/}

\Note 5:1 {Balsazar král}
     \x/Belšasar/ znamená \uv{\x/Bel/ ochraňuj krále!} 
     Nezaměnit se jménem \x/Beltšasar/, které v Babylóně dostal Daniel (viz <"poznámku" 1:7>n). 
     Nabonidus, \x/Nebúkadnesar/ův zeť, byl posledním vládcem Babylónu. 
     \x/Belšasar/, nejstarší Nabonidův syn, byl ustanoven spoluvládcem společně se svým otcem.
     Byl mu svěřen Babylón, zatímco Nabonidus trávil mnoho času v Arábii.
     Události kapitoly 5 se odehrály v roce 539 př.Kr. (42 let po \x/Nebúkadnesar/ově
     smrti v roce 563 př.Kr., kdy Babylón padl do rukou Peršanů a kdy byl vydán
     edikt, propouštějící Izraelity z otroctví. 

\Note 6:26 {nařízení} 
     Dariův dekret neimplikuje automaticky, že Darius konvertoval od svého
     pohanského polyteismu k víře v Danielova Boha, o nic více, než Kýrova proklamace, že Bůh mu
     dal pokyn poslat Židy domů (<Ezd 1:3-4>, <Iz 44:28>, <Iz 45:4>).

\Note 7:1-12:13 {}={Danielovy vize} 
     V těchto kapitolách Daniel opouští historické vyprávění
     a zaznamenává své vize. Tyto vize navazují na předchozích šest kapitol dvěma hlavními tématy: 
     1)~Hospodin, Bůh Izraele, je svrchovaný Pán nade všemi národy a 
     2)~Daniel, nekompromisní Boží prorok, je spolehlivě důvěryhodný. Tyto kapitoly připravují exulanty na dlouhé čekání na plné znovuobnovení Izraele, jakož i na zkoušky a utrpení pod
        nadvládou cizích mocností. Jsou také Božímu lidu povzbuzením, aby se nevzdával naděje,
        že Boží království jednou přijde učinit všemu trápení konec. Daniel se dotýká čtyř
        hlavních témat: 1)~čtyři \x/zvířata/ (<7:1-28>),  
                        2)~beran a kozel (<8:1-27>),
                        3)~\uv{sedmdesát týdnů} (<9:1-27>) a 
                        4)~budoucnost Božího lidu (<10:1>--<12:13>).   

\Note 7:1-28 {}={Vize čtyř \x/zvířat/}
     Danielův sen o čtyřech šelmách zachycuje historii střídání cizích
     království, které Izrael utiskovaly, až do doby, kdy jejich pozemská vláda byla dána 
     \uv{\x/lidu svatých Nejvyššího/}.

\Note 7:3 {moře} 
     Není zřejmé, zda je míněno nějaké konkrétní moře (snad Středozemní?). Nicméně
     lze mít za to, že moře symbolizuje chaotický neklid, charakteristický pro hříšné národy,
     okupující Izrael. Viz interpretaci ve verši <17> a v <Iz 17:12-13> a <Iz 57:20>. 

\Note 7:3 {čtyři šelmy}
    Čtyři \x/zvířata/ reprezentují čtyři království (vv. <17> a <23>).
    Spojitost  s \x/Nebúkadnesarovou/ vizí sochy v kapitole 2 je zřejmá. Pro jejich identifikaci viz náčrt 
    {\it Danielovy vize\/} na str.~\pgref[danielovavize]. 

\Note 7:4 {lvu}={lev ... orličí křídla} 
     Lev s orlími křídly symbolizuje Babylón (srv.~<Jr 50:44>, <Ez 17:3>).
     Okřídlení lvi byli běžné babylónské artefakty, často umisťované u vchodů významných veřejných budov.      {\bf \x/oškubána/ ... \x/lidské srdce/} Snad odkaz na \x/Nebúkadnesarovu/
     proměnu a navrácení do lidské společnosti po sedmiletém ponížení nepříčetností
     <Da 4:31-34>).

\Note 7:5 {nedvědu}={nedvěd ... k jedné straně ... tři žebra} 
     Médo-perské království je symbolizováno šelmou s nenasytnou žravostí. Vztyčená
     strana může reprezentovat nadřazenou pozici Persie. Tři žebra pravděpodobně znamenají
     vítězství Persie nad Lydií (546 př.Kr.), Babylónem (539 př.Kr.) a Egyptem (525 př.Kr.).
     Viz <"poznámku" 8:3>n.

\Note 7:6 {pardovi}={pard ... čtyři ptačí křídla ... čtyřhlavé}
     Řecko je symbolizováno \x/pard/em, proslulým svou rychlostí.
     Alexandr Veliký (356--323 př.Kr.) dobyl Persii velmi rapidně.
     Střetl se s Peršany ve třech velkých bitvách:
     1) bitvou u řeky Gráníkos (334 př.Kr.) získal vstup do Malé Asie; 
     2) bitva u Issu (333 př.Kr.) mu umožnila okupovat Sýrii, Kenaán a Egypt; 
     3) v~bitvě u~Gaugamél porazil perskou armádu definitivně a otevřel si cestu do Indie.
        Viz též <Da 8:5-8>. Krátce po jeho předčasné smrti (ve věku 33 let) se říše, kterou
        vytvořil, rozpadla na čtyři části: v Makedonii vládl Kassandros, v Thrákii a Malé Asii
        Lýsimachos, v Sýrii Seleukos a v Egyptě Ptolemaios.

\Note 7:7 {šelma čtvrtá}
     Historie nás učí, že tato neidentifikované zvíře je Řím --- impérium, které postupně
     asimilovalo různé části rozděleného řeckého království. 
     {\bf deset rohů} Znamenají deset římských králů (viz v. <24>). Není zřejmé, zda následují po
     sobě nebo vládnou současně. Pro spekulativní domněnku, že mají znamenat druhou fázi čtvrtého
     království, \uv{oživenou říši římskou} posledních dnů, však jednoznačně přesvědčivý textový
     důkaz neexistuje. 

\Note 7:8 {roh poslední}={roh poslední ... tři z dřívějších rohů byly před ním vyvráceny}
     Deset rohů časově předchází \uv{malému,} který vyvrátí tři z~nich. Je to další fáze
     čtvrtého království. Mnozí mají za to, že malý roh symbolizuje vzestup antikrista 
     <2Te 2:3-4>. Pokud je tomu tak, pak je toto první zmínka o antikristu v Písmu.  

\Note 7:8 {oči podobné očím lidským}{oči podobné očím lidským ... a ústa}
     Metafora naznačuje, že roh reprezentuje spíše člověka než království. 
     
\Note 7:9 {Starý dnů} Jediný výskyt v Bibli je v této kapitole (srv. <13>, <22>). Podobný výraz se objevuje v ugaritských textech k označení velkého Boha  {\em El}. Zde je použit jako označení pro Boha, který zasedl k soudu, a implikuje, že je věčný a  panuje od pradávna.

\Note 7:9 {roucho}={roucho ... vlasové} Ačkoliv se Bůh Danielovi zjevil v neskutečné slávě, přesto to bylo v podobě rozpoznatelné jako lidská.

\Note 7:9 {trůn}={trůn ... kola} Vyobrazení Božího trůnu koreluje s vizí proroka Ezechiele (<Ez 1:15-28>).
      Nebeský trůn je zobrazen s koly (podobně jako v památkách jiných národů z oné doby) --- jako královský válečný vůz. Podobný motiv se skrývá za ohnivým sloupem, který vedl Izrael během Exodu (<Ex 13:21-22>).

\Note 7:13 {Synu člověka} Tento termín může znamenat jednoduše \uv{člověk}. Hebrejský
   ekvivalent tohoto aramejského výrazu je v <Da 8:17>  použit pro Daniele, stejně jako pro jeho současníka
   Ezechiele v <Ez 2:1>, <Ez 2:3>, <Ez 2:6>. 
   Daniel je jedním z prvních (ne\discretionary{-}{-}{-}li
   vůbec první), kdo používá toto spojení.  Pozdější židovská mezizákonní apokalyptická
   literatura navazuje na tuto pasáž a vykresluje \uv{syna člověka} jako nadpřirozenou bytost,
   přinášející nebeskou moc na Zem. Daniel viděl {\em podobného Synu člověka}, tedy někoho srovnatelného s člověkem, a přesto výrazně odlišného (<14>). V evangeliích  je výraz \uv{Syn člověka} používán ve vztahu ke Kristu (69 výskytů v synoptických evangeliích a 12 v Janově). Ježíš sám sebe nejčastěji označoval právě tímto titulem.
   
\Note 8:1 {Léta třetího kralování Balsazara}
     Viz <"poznámku" 5:1>n. Není jednoznačně jisté, zda \x/Belšasarova/
     spoluvláda s Nabonidem začala zároveň s Nabonidovým nástupem (556 př.Kr.) nebo o něco
     později. Ať tak či onak, události této (a osmé) kapitoly je nutno chronologicky umístit
     mezi události kapitol 4 a 5. 

\Note 9:2 {Jeremiáš}={Jeremiáš ... sedmdesátého léta} Viz <Jr 25:11>.

\Note 9:3 {modlitbou}={modlitbou, v postu, žíni a popele} Daniel byl zděšen, protože věděl, že Izrael byl 70 let v zajetí za trest pro své hříchy, ale ani po 70 letech se od svých hříchů neodvrátil. Srv. <"pozn. k" 11>.

\Note 9:11 {Mojžíše} Danielovi bylo jasné, že když si Izrael nevzal ponaučení ze sedmdesátiletého otroctví, čeká ho sedminásobný trest, jak píše Mojžíš (<Lv 26:18,21,24,28>). Národ bude sloužit cizincům 490 dalších let. Historie nás učí, že tato lhůta byla Boží milostí o něco zkrácena (srv. <Jr 18:8>).
\dopsat

\Note 9:26 {Mesiáš}={Mesiáš ... svatyni} Podle tohoto proroctví bude Mesiáš zabit před zničením jeruzalémského chrámu.
Druhý chrám, který zbudoval Herodes v roce 20 př.Kr., byl srovnán se zemí Římany v roce 70 po Kr. a od té doby nebyl nikdy obnoven. Na jeho místě dnes stojí mešita; není žádá naděje, že by mohl být znovu zbudován, aby před jeho dalším zničením mohl být zabit Mesiáš. Příchod Mesiáše musíme hledat mezi 20 př.Kr. a 70 po Kr.

\Note 10:12 {přiložil srdce své, abys rozuměl} Danielova moudrost nebyla náhodná; usiloval o ni celým srdcem (srv. <Jr 29:16>). 

\Note 10:13 {kníže království Perského} Teritoriální démon, okupující Persii. V anglických Biblích nese označení ``Prince of Persia.'' Populární videohra a film téhož jména jsou typickou ukázkou zpohanštění kultury, kdysi křesťanské: Lze je považovat za oslavu tohoto prastarého démona.

\Note 10:13 {jedenmecítma dnů} (Dvacet jedna dnů) Když se modlíme, uvádíme do pohybu souvislosti v neviditelném světě, o jakých většinou nemáme ani sebemenší tušení.







\endinput
