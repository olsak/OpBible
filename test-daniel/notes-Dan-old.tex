
\CommentedBook{Dan}

\Note 1:1 {} Nejznámější verš v Bibli je <Jn 3:16>.  

\Note 1:7  {Beltšasar ... \v/Sidrach/ ... \v/Mesak/ ... \v/Abednego/}
     Přesný význam těchto jmen je předmětem diskusí. Převažují tyto názory: 
     \v/Baltazar/: {\it \v/Bel/} [jiné jméno pro Marduka, hlavního boha babylónského panteonu]
     {\em chraň jeho život} nebo {\em Paní, ochraňuj krále}; 
     \v/Sidrach/: {\em Velice se bojím (Boha)} nebo {\em Přikázání Aku} [sumérský měsíční bůh];
     \v/Mesak/:   {\em Jsem bezvýznamný} nebo {\em Kdo je to, co Aku?};
     \v/Abednego/: {\em Služebník zářícího.}

\Note 4:13 {srdce zvířecí}
     \v/Nebukadnesar/  byl postižen mentální poruchou zvanou lykantropie 
     (z řeckého {\em lukos \ {\v/lukos/} --- vlk} a {\em anthropos {\v/anthropos/} --- člověk}), 
     při níž  se člověk chová jako vlk nebo i jiné zvíře. Viz též <P 4:30>. 

\Note 4:13 {sedm let} 
     Sedm období neurčené délky (srv. vv. <20> a <22>). Většina interpretů se shoduje na závěru, že 
     \v/léto/ znamená jeden rok. Verš <30> naznačuje, že doba byla delší než den, týden či měsíc.

\Note 4:30 {pojídal rostliny jako dobytek}
     Vzhledem k tomu, že  se \v/Nebúkadnesar/ projevoval rysy charakteristickými pro býložravce,
     je jeho mentální porucha někdy nazývána {\it boantropií.\/}

\Note 5:1 {Král Belšasar}
     slovo \v/Belšasar/ znamená \uv{\v/Bel/ ochraňuj krále!} 
     Nezaměnit se jménem \v/Beltšasar/, které v Babylóně dostal Daniel (viz <"poznámku" 1:7>n). 
     Nabonidus, \v/Nebúkadnesarův/ zeť, byl posledním vládcem Babylónu. 
     \v/Belšasar/, nejstarší Nabonidův syn, byl ustanoven spoluvládcem společně se svým otcem.
     Byl mu svěřen Babylón, zatímco Nabonidus trávil mnoho času v Arábii.
     Události kapitoly 5 se odehrály v roce 539 př.Kr. (42 let po \v/Nebúkadnesarově/
     smrti v roce 563 př.Kr., kdy Babylón padl do rukou Peršanů a kdy byl vydán
     edikt, propouštějící Izraelity z otroctví. 

\Note 6:26 {nařízení} 
     Dariův dekret neimplikuje automaticky, že Darius konvertoval od svého
     pohanského polyteismu k víře v Danielova Boha, o nic více, než Kýrova proklamace, že Bůh mu
     dal pokyn poslat Židy domů (<Ezd 1:3-4>, <Iz 44:28>, <Iz 45:4>).

\Note 7:1--12:13 {} % to neodkazuje přesně na jedno místo výchozího textu, není to spíše glosa? 
   {\em Danielovy vize} 
     V těchto kapitolách Daniel opouští historické vyprávění
     a zaznamenává své vize. Tyto vize navazují na předchozích šest kapitol dvěma hlavními tématy: 
     1)~Hospodin, Bůh Izraele, je svrchovaný Pán nade všemi národy a 
     2)~Daniel, nekompromisní Boží prorok, je spolehlivě důvěryhodný. Tyto kapitoly připravují exulanty
        na dlouhé čekání na plné znovuobnovení Izraele, jakož i na zkoušky a utrpení pod
        nadvládou cizích mocností. Jsou také Božímu lidu povzbuzením, aby se nevzdával naděje,
        že Boží království jednou přijde učinit všemu trápení konec. Daniel se dotýká čtyř
        hlavních témat: 1)~čtyři \v/zvířata/ (<7:1-28>),  
                        2)~beran a kozel (<8:1-27>),
                        3)~\uv{sedmdesát týdnů} (<9:1-27>) a 
                        4)~budoucnost Božího lidu (<10:1>--<12:13>).   

\Note 7:1--28 {}
    {\em Vize čtyř \v/zvířat/}
     Danielův sen o čtyřech šelmách zachycuje historii střídání cizích
     království, které Izrael utiskovaly, až do doby, kdy jejich pozemská vláda byla dána 
     \uv{\v/lidu svatých Nejvyššího/}.

\Note 7:1 {V prvním roce ... Belšasara} % odkazuje na "v prvním roce" nebo na "Belšasara"?
     Viz <"poznámku" 5:1>n. Není jednoznačně jisté, zda \v/Belšasarova/
     spoluvláda s Nabonidem začala zároveň s Nabonidovým nástupem (556 př.Kr.) nebo o něco
     později. Ať tak či onak, události této (a osmé) kapitoly je nutno chronologicky umístit
     mezi události kapitol 4 a 5. 

\Note 7:2 {moře} 
     Není zřejmé, zda je míněno nějaké konkrétní moře (snad Středozemní?). Nicméně
     lze mít za to, že moře symbolizuje chaotický neklid, charakteristický pro hříšné národy,
     okupující Izrael. Viz interpretaci ve verši <17> a v <Iz 17:12-13> a <Iz 57:20>. 

\Note 7:3 {čtyři zvířata}
    Čtyři \v/zvířata/ reprezentují čtyři království (vv. <17> a <23>).
    Spojitost  s \v/Nebúkadnesarovou/ vizí sochy v kapitole 2 je zřejmá. Pro jejich identifikaci viz náčrt 
    {\it Danielovy vize\/} na str.~\pgref[danielovavize]. 

\Note 7:4 {lev ... orlí křídla} 
     Lev s orlími křídly symbolizuje Babylón (srv.~<Jr 50:44>, <Ez 17:3>).
     Okřídlení lvi byli běžné babylónské artefakty, často umisťované u vchodů významných veřejných budov. 
     {\bf \v/oškubána/ ... \v/lidské srdce/} Snad odkaz na \v/Nebúkadnesarovu/
     proměnu a navrácení do lidské společnosti po sedmiletém ponížení nepříčetností
     <Da 4:31-34>).
|
\Note 7:5 {medvěd ... k jedné straně ... tři žebra} 
     médo-perské království je symbolizováno šelmou s nenasytnou žravostí. Vztyčená
     strana může reprezentovat nadřazenou pozici Persie. Tři žebra pravděpodobně znamenají
     vítězství Persie nad Lydií (546 př.Kr.), Babylónem (539 př.Kr.) a Egyptem (525 př.Kr.).
     Viz <"poznámku" 8:3>n.

\Note 7:6 {jako levhart ... čtyři ptačí křídla ... čtyřhlavé}
     Řecko je symbolizováno \v/levhartem/, proslulým svou rychlostí.
     Alexandr Veliký (356--323 př.Kr.) dobyl Persii velmi rapidně.
     Střetl se s Peršany ve třech velkých bitvách:
     1) bitvou u řeky Gráníkos (334 př.Kr.) získal vstup do Malé Asie; 
     2) bitva u Issu (333 př.Kr.) mu umožnila okupovat Sýrii, Kenaán a Egypt; 
     3) v~bitvě u~Gaugamél porazil perskou armádu definitivně a otevřel si cestu do Indie.
        Viz též <Da 8:5-8>. Krátce po jeho předčasné smrti (ve věku 33 let) se říše, kterou
        vytvořil, rozpadla na čtyři části: v Makedonii vládl Kassandros, v Thrákii a Malé Asii
        Lýsimachos, v Sýrii Seleukos a v Egyptě Ptolemaios.

\Note 7:7 {čtvrté zvíře, strašné, příšerné a mimořádně mocné}
     Historie nás učí, že tato neidentifikované zvíře je Řím --- impérium, které postupně
     asimilovalo různé části rozděleného řeckého království. 
     {\bf deset rohů} Znamenají deset římských králů (viz v. <24>). Není zřejmé, zda následují po
     sobě nebo vládnou současně. Pro spekulativní domněnku, že mají znamenat druhou fázi čtvrtého
     království, \uv{oživenou říši římskou} posledních dnů, však jednoznačně přesvědčivý textový
     důkaz neexistuje. 

\Note 7:8 {další malý roh ... tři z dřívějších rohů byly před ním vyvráceny}
     Deset rohů časově předchází \uv{malému,} který vyvrátí tři z~nich. Je to další fáze
     čtvrtého království. Mnozí mají za to, že malý roh symbolizuje vzestup antikrista 
     <2Te 2:3-4>. Pokud je tomu tak, pak je toto první zmínka o antikristu v Písmu.  

\Note 7:8 {oči jako oči lidské a ústa, která mluvila troufale}
     Metafora naznačuje, že roh reprezentuje spíše člověka než království. 

\endinput

Komentáře k notaci
------------------

Soubor s poznámkami se bude jmenovat notes-Dan.tex nebo podobně a poznámky
budou ke každé knize ve zvláštním souboru.

Na začátku souboru bude \CommentedBook{zkratka} odkazující na konkrétní
knihu bible, na které se váží poznámky.

Jednotlivá poznámka má syntax:
\Note kapitola:verš {upřesnění} text prázdný řádek.

Pozámka končí prázdným řádkem. Předpokládám, že nebude mít poznámka více
odstavců, když ano, tak se dá použít \endgraf.

\v/výraz/ je variantní výraz definovaný pomocí \vdef /výraz/, (původně \vardef)
pro různé překlady může být různý. Samotný "výraz" by měl být text podle
jednoho z překladů (zvolil jsem CEP) a použije se jiný výraz jen za
předpokladu, že není aktuálně zpracovávaný CEP. Pokud v souboru
variants.tex není výraz definován a je třeba zpracovat něco jiného než CEP,
objeví se varování o nedefinovaném výrazu a použije se výraz z CEP.

V parametru \Note je {upřesnění}. To může být prázdné, pak poznámka odkazuje
na začátek verše. Je-li {upřesnění} nějaký výraz, pak TeX vyhledá tentýž
výraz ve verši (první jeho výskyt) a poznámka odkazuje na toto místo.

Pokud {upřesnění} obsahuje tři tečky (přesněji mezera...mezera), pak odkazovaný
výraz se hledá jen ten první před prvními třemi tečkami. Ostatní výrazy ze
třemi tečkami na výchozí biblický text nijak nenavazují.

V textu je možné psát ... místo \dots\ , LuaTeX to v případě, že je to
potřeba vytisknout, nahradí odpovídající typografickou elisou.

Pokud je {upřesnění} v prvním pádě a ve skutečnosti je ve výchozím textu
použit jiný pád, je třeba přidat varianty s různými pády a výrazy do
souboru variants.tex, jinak to TeX nenajde.

Pokud TeX nenajde {upřesnění}, napíše o tom varování a odkáže na začátek
poznámky.

{upřesnění} se v poznámce vždy vytiskne jako první výraz (dle typografického
zadání například tučně). Před tiskem TeX zkontroluje, zda není ve variants.tex 
k {upřesnění} varianta aktuálně použitého překladu. Když není, tak vytiskne 
"upřesnění" přesně tak, jak je. V tomto případě bez varování o chybějící
variantě.

Pokud chceme mít v textu odkaz na verš, navrhuji jej označit takto:

<30> aktivní odkaz "30" na daný verš aktuální kapitoly a knihy.
<3:30> aktivní odkaz "3:30" na daný verš a kapitolu aktuální knihy.
<Gn 3:30> aktivní odkaz "Gn 3:30" na daný verš, kapitolu a knihu.

Má li být součástí aktivního odkazu nějaký text, který předchází 
uvedenému číslu, je možné jej vložit jako první údaj za < do uvozovek:

<"verš" 30> nebo <"v." Gn 3:30>

Má-li odkaz směřovat nikoli na verš, ale na poznámku k verši, článek ke
kapitole nebo glosu, je syntaxe zcela stejná, jen těsně za > je třeba
připojit písmeno n (poznámka, note), a (čánek, article) nebo g (glosa):

Viz <"poznámku" 30>n ... odkazuje na poznámku k verši 30 aktuální kapitoly a knihy. 
Podrobnosti najdete v <"poznámce" Gn 3:30>n ... odkaz na poznámku ke Gn 3:30.

Implementace aktivních odkazů:

\def\currentbook{Dan}
\def\currentchap{7}

\def\bref #1{\ifx"#1\afterfi{\brefA}\else\afterfi{\brefA"#1}\fi}
\def\brefA#1"#2>{\def\brefT{#1#2}\brefR#2 >\futurelet\next\brefB}
\def\brefB{\def\brefC{\brefD v}%
   \ifx\next n\def\brefC{\brefD}\fi
   \ifx\next a\def\brefC{\brefD}\fi
   \ifx\next g\def\brefC{\brefD}\fi
   \brefC
}
\def\brefD #1{[\brefT]--(#1:\brefG)}
\def\brefR #1#2 #3>{\edef\brefG{\ifx^#3^\currentbook-\brefN#1#2 |\else #1#2-\brefN#3|\fi}}
\def\brefN#1 |{\brefM #1\empty:\empty|}
\def\brefM#1:#2\empty#3|{\ifx\relax#2\relax \currentchap:#1\else #1:#2\fi}

\adef<{\bref}

Viz <"poznámku" Gn 3:20>n a <"glosu" 30>g a taky <"v." 2:22>.

\bye
