\load [vlna]
\enquotes \enlang
\margins/1 a4 (,,,)mm
\fixmnotes\right
\fontfam[schola]
\def\TODO{{\Red TODO}}
\everytt{\typosize[9/]\catcode`<=13 } 
\everyintt{\catcode`<=13 }
\def\`{\bgroup \_setverb \toindex}
\def\toindex#1`{\egroup{\tt\Magenta #1}\ea\iindex\ea{\ignoreit#1}}
\def\_printii #1&{\noindent{\tt\bslash#1}\quad}
\def\lookout{\par\noindent {\bf What to look out for:} }
\def\recommended{\par\noindent {\bf Recommended:} } 
\verbchar`
\catcode`<=13 \def<#1>{{$\langle$\it#1$\rangle$}}
\def\OpBible/{OpBible}
% “ ”
\hyperlinks\Blue\Blue

{\nopagenumbers

\tit \OpBible/

\centerline{\typosize [16/]\bf Automatic Study Bible Typesetting Using \TeX/}
\bigskip
\centerline{\bi Version 1.0}
\bigskip
\centerline{\setfontsize{at15pt}\it Petr Olšák, Šimon Konečný}
\bigskip\bigskip
\centerline{\it Revised \the\day. \the\month. \the\year}
\bigskip
\centerline{https://www.overleaf.com/project/62e3832c89a3bfe389d592ba}

\vfill
{\it
\rightline{It is too small a thing for you to be my servant}
\rightline{to restore the tribes of Jacob}
\rightline{and bring back those of Israel I have kept}
\rightline{I will also make you a light for the Gentiles,}
\rightline{that my salvation may reach to the ends of the earth.}
\smallskip
\rightline{Isaiah 49:6} %CEP
}\eject


The \OpBible/ macro collection may be freely copied, distributed and used in accordance with GNU
General Public License (Version 2, 1991).

You can port parts of this software into your own macros and/or make them part of other
packages, but the package, however different from the original distribution, must not be named \OpBible/.

Adaptations of specific implementations (e.g.\ fonts) are considered additional files and
their inclusion does not violate the license.


\vfill\eject

\nonum\sec[preface] Preface


The verse from Isaiah 49 on the title page is inspiring: It would be a small thing to write and typeset using \TeX/  just one Bible commentary  as one's life's work. The \OpBible/ offers the opportunity to write a commentary on any Bible version to anyone anywhere; once the annotation apparatus is complete, the study Bible is typeset. 

Šimon Konečný's long-standing vision of using \TeX/ to automatically typeset study Bibles came to fruition through a collaboration with Petr Olšák.  Starting from the realistic assumption that
those who will use OpBible the most will also be those who have the least experience with \TeX/, we tried to make the usage as simple as possible.

However, there is no point in pretending that typesetting a study Bible is something trivial.  
To claim that it is a little tricky is a bit understatement;
it is rather quite a challenge for any programmer.

Our intention to allow the production of a single commentary apparatus to gop with multiple Bible translations simultaneously was
further complicated by the fact that different versions of the Bible have different verse numbers in some places, differently broken paragraphs, 
different structures of poetic passages, different placement of headings added by the translators; that  some translations have completely different book titles (and hence abbreviations),
that file names cannot have diacritics, but in book references can, and countless other such oddities.

All the problems we encountered have been overcome and everything works just fine to our satisfaction.  The resulting collection of tools is quite complicated,
therefore  the definitions usage is irritatingly sensitive to transgressions against \TeX/ syntax.

That's why \OpBible/ includes tools to make it easier to find where typos, forgotten parentheses, etc. have been left; and in this documentation,
wherever we have seen fit, we add paragraphs entitled {\bf What to look out for} and {\bf Recommended}.




\vfill 

\rightline{Petr Olšák}
\rightline{Šimon Konečný}
\eject








%\sec What is the \OpBible/?
%
%\noindent
%\OpBible/ is a set of tools (macros and other support software) for processing biblical texts. The main goal is to link the text of the Bible with the commentary notes.
%%according to the required criteria described below.
%The result after processing by \TeX/ is a study Bible, i.e. a Bible printed together with annotations and typeset automatically by \TeX/.
%%The %%PDF file is also richly hyperlinked.

%\medskip
 %%\centerline{\picw=15cm \inspic daniel.png }
%\medskip



\notoc\nonum\sec Contents
\maketoc
\vfil\break



%\nonum\notoc
\sec What makes the \OpBible/ specific?

The main advantages of the \OpBible/ over traditional typesetting are:

\begitems
* {\bf Price:} \begitems
  * The software itself is free as public domain under an Open Source license (see Preface). 
  * More importantly, however, is the price which
  you'll save on a typesetter you'd otherwise have to pay for several years to hand-typeset the Study Bible.
        \enditems
* {\bf Time:} Typesetting an entire Study Bible takes minutes, not years. To provide the full text of a Study Bible  that breaks up the pages so that the notes end up on the same pages with
  the verses they're commenting on, it takes a system with a quad-core Ryzen 3 processor
  ten minutes. If you're working on a particular book and only processing one, it's usually a matter of seconds.
* {\bf Flexibility:} The result is not one single version of the Bible in which you cannot 
add  any changes. If you decide to edit some notes, delete some or add more, etc.,
  you have this new revised version immediately typeset and ready for print.
* {\bf Variability:} The result may not be
  just one Bible. The notes can be written in a way that allows as many Bibles as there are different translations in a given language to be typeset with the same note apparatus. 
  For example, consider these six English translations of the Bible: 
(1)~Bible in Basic English (BBE); (2)~Jubilee2000; (3)~NETfree; (4)~Updated King James Version (UKJV); (5)~Restored Names King James Version (RNKJV); and (6)~Webster.
Now let's take a brief look at Daniel, chapter 2, where the animals, countries, and other stuff of Daniel's vision are rendered as follows: 

\begitems
* {\it Bear\/} is the same everywhere but
* {\it Leopard\/} of most tranlations is {\it Tiger\/} in Jubilee2000;
* usual {\it Ram\/} becomes {\it Male sheep,\/} as well as {\it clay\/} changes {\it potter's earth\/} in BBE;
* ordinary {\it brass\/} turns into {\it bronze\/} in NETfree;
* {\it Greece\/} of BBE, Jubilee2000, and NETfree is {\it Grecia\/} in UKJV, RNKJV, and Webster.
\enditems

You can write your commentary in the way that the result would be all six translations at the same time, with no need to modify the notes file, while the notes (or charts, for that matter), will display the phrases as they appear in your current Bible. See sample book of Daniel, chapter two, the chart of Daniel's visions.

In English, there are over 450 different translations of the Bible.
Most of them are not being used anymore, but still, 
you can choose from \ulink[https://bibleanswers.study/about-the-bible/types-of-bible-versions]{more than sixty} various English versions. 


* {\bf Interactive output format}:
  The result of the processing is a PDF file that is richly linked with active links. 
  These are an invaluable aid for proof-reading  before submission to the print (see below).

* {\bf Precision:} \begitems
  * The average study Bible contains around 20,000 notes and within them around 80,000
  references to Bible verses and other notes. The notes are written by humans; it is inevitable  that  they will contain errors from oversight or typos. It is practically beyond the human power 
  to find and correct  all of them. It is, however, within the power of A machine. \OpBible/ prints warning messages,  if, for example, a note refers to a non-existent note or verse.
*  It  checks for the accurateness of references to verses and notes that do exist (thus they do 
   not trigger an error message),
    but for some reason they are not the right ones.
    The control is made possible by the fact that all references are active links that display the immediate context when the cursor hovers over
   the linked location (if you are viewing the PDF with the correct version of Evince, for example). And when clicked on, it jumps straight there. 
  * The phrase that the note comments on is highlighted with a different font (e.g. bold).
    However, it will also searched for in the Bible text inside the verse in question and the pages will be broken in the way that places  the comment is on the same page as its phrase, not just by verse number.
    In case that there are  that two notes commenting on two different words in the same verse, and that    the page break will come between those words,  the notes will follow
   the phrases on their pages. %See \ref~[break-in-verse] %zatím není v en
   * The Bible texts are prepared (for example, by downloading them from the Internet) in separate files and there is no need for them to be modified in any way. All notes on them and other typesetting instructions are written in separate files. Then \TeX/ will put everything together.
    \enditems
\enditems

%{\bf What to watch out for:} 
\lookout
Although the intent of this program is to spread God's Word and the Good news contained therein, the \OpBible/ itself {\it forgives nothing!} 
After all, what'd you expect, it's a software. (Go to God for forgiveness.) As you'll read below, the \OpBible/ loads all notes for one entire book of the Bible into memory at a time and only then  begins to digest these notes and piece them together into pages of their respective verses. If you happen to make a mistake in \TeX/ syntax (e.g., forget a closing bracket, etc.), \TeX/ will see the error somewhere in the middle of this matching procedure, and this will also result in a cluttered error message, in which it is highly likely that you will get lost and not find your way around.

The \OpBible/~ anticipates the occurrence of situations like this and offers useful tools to help you out of similar predicaments, but you'd better be careful and consistent. You won't go wrong with the first written note, to go through all the translations you plan to use to make sure everything works as it should.  After that you only need to work with one of them, but it is advisable to run \TeX\ every time  after you finish each new note, so by its protests  you can easily find the one you have a mistake~in.

\recommended Use the Linux operating system. \OpBible/, as a macro file for \TeX, will run on any system with the same results, but the Linux installation has
several not-to-be-missed advantages. Among the most significant of these is the Evince PDF viewer, whose newer (as of 2023) version can display the immediate context of a hyperlinked
reference by simply hovering the cursor over it without having to click on it. This is an invaluable aid for checking the accuracy of links to Bible passages or notes.
Other systems do not (yet) offer this facility. While the hyperlinks in these systems do jump to the appropriate location when clicked, they no longer hit back to where the jump signal was sent from, which is tedious. In this documentation, we assume a Linux installation.

And of course, we recommend -- or rather, consider it a prerequisite -- at least basic literacy in \TeX.
If you don't have any experience with \TeX, try starting with 
\ulink[https://www.ctan.org/tex-archive/info/gentle]{here}\fnote{This text is somewhat old and refers about DVI output of \TeX/. Nowadays, modern \TeX/ ouputs directly to the PDF format.}. 
The time invested in this preliminary education will not be wasted; the more you understand \TeX/, the better (and more enjoyably) you will find writing your Bible notes.


\sec[Installation] What we need to run \OpBible/

It is necessary to have some kind of text editor that does not leave any hidden formatting information in the text (designed for editing programs, for example). It is up to the user what kind of editor suits them.
The ideal editor is one that recognizes the programming language by its source file's extension and colors words according to its syntax. E.g.\ Vi, Emacs, ... 
\ulink[https://en.wikipedia.org/wiki/List_of_text_editors]{choose}, whichever would suit you. 


You also need to have a \TeX/ distribution with the \LuaTeX/ program and \OpTeX/ default macros, and finally you need some kind of output (i.e.\ PDF files) viewer (we recommend the last version of Evince).

It isn't important on which operating system you will run this, but we recommend 
\ulink[https://www.hostinger.com/tutorials/best-linux-distro]{Linux}.

It is also possible to work in online mode without installing anything, see section~\ref[OverLeaf].

\secc[TeXlive] \TeX/ at the local machine

We recommend the latest \ulink[https://tug.org/texlive/]{\TeX/live} distribution.
It can be installed on any operating system directly from the web. It is also included in common Linux distributions. 

\TeX/live contains the program \LuaTeX/, which will process your input files and produce output PDFs. It also includes the default macro set \ulink[http://petr.olsak.net/optex/]{\OpTeX/} that the \OpBible/ macros need, and extends it with the options described in this manual. The default set of macros (often called format) defines how documents are markup and controls formatting. The \TeX/ distributions include other default macro sets. The best known is probably \LaTeX/, but it does not need \OpBible/.

\secc[OverLeaf] Variant: remote access on Overleaf.com

Overleaf is a web-based environment for shared preparation of \TeX/ documents, perhaps by multiple users. You do not need to have \TeX/live isntalated and can run it online via \ulink[https://www.overleaf.com]{Overleaf}. \TODO The \ulink[??]{default project} containing files for processing by \OpBible/ is also available. As an Overleaf user, you can copy (clone) it to your project and continue working there. However, working with the \TeX/ distribution directly on your computer is much faster and more convenient.

%\secc[evince] Evince

%`flatpak` : flatpak run org.gnome.Evince

%The full text of the Bible is available on the Internet in various translations and English

\sec Running \TeX/

If you have a \TeX/ distribution installed on your machine (for example, \TeX/live
2022 or newer) %; if not, see the \ref~[installation] chapter)
and if you have a command line available on your system, then you can run \TeX/ using:
\begtt
optex main.tex
\endtt
where `main.tex` is the name of the main file (it can have a different name). 
The `optex` command runs \LuaTeX/ with the \OpTeX/ macros. The result of the processing is a file 
`main.pdf` and the processing message is in the `log` file `main.log`.

You can test if this works for you (without \OpBible/ macros for now). Create a simple `main.tex` file in a text editor with this content:

\begtt
\fontfam[lm]
Hello world!
\bye
\endtt
and run `optex main.tex`. View the resulting PDF file `main.pdf` with a PDF viewer.


\sec Structure of files processed by \TeX/

\secc[main] Main file with information about all other files

The main file is the file that is submitted to \TeX/ first. For example, it is
listed on the command line to start \TeX/u. It contains information about what
other files to be read by \TeX/. Finally, \TeX/ will create a PDF file of the same
name as the main file name.

The main file for RUNNING \OpBible/ (for example, `main.tex`) might look something like this:

\begtt
\load[opbible] % of the OpBible macro
\enlang % initializing English hyphenation patterns

% Translation variants:
\def\tmark {BBE} %  Bible in Basic English
%\def\tmark {Jubilee2000} % Jubilee 2000
%\def\tmark {NETfree} % New English Tranlation
%\def\tmark {UKJV} % Updated King James Version
%\def\tmark {RNKJV} % Restored Names King James Version
%\def\tmark {Webster} % Webster Bible

\input {Cze-vars.tex} % Phrase declarations for different translation options
\BookTitle Gen  Gn {První Mojžíšova (Genesis)}
\BookTitle Exod Ex {Druhá Mojžíšova (Exodus)}
\BookTitle Lev  Lv {Třetí Mojžíšova (Leviticus)}
\BookTitle Num  Nu {Čtvrtá Možíšova (Numeri)}
\BookTitle Deut Dt {Pátá Mojžíšova (Deuteronomium)}
\BookTitle Josh Joz {Jozue}
\BookTitle Judg Sd {Soudců}
\BookTitle Ruth Rt {Rút}
\BookTitle 1Sam 1S {První Samuelova}
\BookTitle 2Sam 2S {Druhá Samuelova}
\BookTitle 1Kgs 1Kr {První Královská}
\BookTitle 2Kgs 2Kr {Druhá Královská}
\BookTitle 1Chr 1Pa {První Paralipomenon (1. Letopisů)}
\BookTitle 2Chr 2Pa {Druhá Paralipomenon (2. Letopisů)}
\BookTitle Ezra Ezd {Ezdráš}
\BookTitle Neh  Neh {Nehemjáš}
\BookTitle Esth Est {Ester}
\BookTitle Job  Jb {Jób}
\BookTitle Ps   Ž {Žalmy}
\BookTitle Prov Př {Přísloví}
\BookTitle Eccl Kaz {Kazatel}
\BookTitle Song Pís {Píseň písní}
\BookTitle Isa  Iz {Izajáš}
\BookTitle Jer  Jr {Jeremjáš}
\BookTitle Lam  Pl {Pláč}
\BookTitle Ezek Ez {Ezechiel}
\BookTitle Dan  Da {Daniel}
\BookTitle Hos  Oz {Ozeáš}
\BookTitle Joel Jl {Jóel}
\BookTitle Amos Am {Ámos}
\BookTitle Obad Abd {Abdijáš}
\BookTitle Jonah Jon {Jonáš}
\BookTitle Mic  Mi {Micheáš}
\BookTitle Nah  Na {Nahum}
\BookTitle Hab  Abk {Abakuk}
\BookTitle Zeph Sf {Sofonjáš}
\BookTitle Hag  Ag {Ageus}
\BookTitle Zech Za {Zacharjáš}
\BookTitle Mal  Mal {Malachiáš}
\BookTitle Matt Mt {Matouš}
\BookTitle Mark Mk {Marek}
\BookTitle Luke Lk {Lukáš}
\BookTitle John Jn {Jan}
\BookTitle Acts Sk {Skutky apoštolské}
\BookTitle Rom  Ř {Římanům}
\BookTitle 1Cor 1Kor {První list Korintským}
\BookTitle 2Cor 2Kor {Druhý list Korintským}
\BookTitle Gal  Ga {Galatským}
\BookTitle Eph  Ef {Efezským}
\BookTitle Phil Fp {Filipským}
\BookTitle Col  Kol {Koloským}
\BookTitle 1Thess 1Te {První list Tesalonickým}
\BookTitle 2Thess 2Te {Druhý list Tesalonickým}
\BookTitle 1Tim 1Tm {První list Timoteovi}
\BookTitle 2Tim 2Tm {Druhý list Timoteovi}
\BookTitle Titus Tt {Titovi}
\BookTitle Phlm  Fm {Filemonovi}
\BookTitle Heb   Žd {Židům}
\BookTitle Jas   Jk {List Jakubův}
\BookTitle 1Pet  1Pt {První list Petrův}
\BookTitle 2Pet  2Pt {Druhý list Petrův}
\BookTitle 1John 1Jn {První list Janův}
\BookTitle 2John 2Jn {Druhý list Janův}
\BookTitle 3John 3Jn {Třetí list Janův}
\BookTitle Jude  Ju {List Judův}
\BookTitle Rev   Zj {Zjevení Janovo}     

\BookException Ž   {\def\amark{Z}}
\BookException Př  {\def\amark{Pr}}
\BookException Pís {\def\amark{Pis}}
\BookException Ř   {\def\amark{R}}
\BookException Žd  {\def\amark{Zd}}

\def\nochapbooks{Abd Fm 2Jn 3Jn Ju}
 % Book titles and bookmarks \amark

\def\txsfile {sources/Eng\tmark-\amark.txs} % Location of txs files
\def\fmtfile {formats/fmt-Eng\tmark-\amark.tex} % Location of fmt files
\def\notesfile {notes/notes-\amark.tex} % Location of notes files
\def\introfile {others/intro-\amark.tex} % Location of book introduction files
\def\articlefile {others/articles-\amark.tex} % Location of article files

\def\printedbooks {%
 Gen Exod Lev Num Deut Josh Judg Ruth 1Sam 2Sam 1Kgs 2Kgs 1Chr 2Chr Ezra Neh Esth Job Ps 
 Prov Eccl Song Isa Jer Lam Ezek Dan Hos Joel Amos Obad Jonah Mic Nah Hab Zeph Hag Zech Mal    
 Matt Mark Luke John Acts Rom 1Cor 2Cor Gal Eph Phil Col 1Thess 2Thess 1Tim 2Tim Titus  
 Phlm Heb Jas 1Pet 2Pet 1John 2John 3John Jude Rev
}

\processbooks % Generates document with books declared in \printedbooks
\bye
\endtt

Now let's see what each of these lines does and which ones will require modification on your part for the specific needs of your project.

Using `\load[opbible]`,  the \TeX/ loads macros of the \OpBible/ package. This is the most important program that takes care of the typesetting.

The \`\enlang` command sets the English word division patterns, so it assumes English
text. The `en` is an ISO language abbreviation, you can use other languages:
`\cslang` for Czech, `\delang` for German, `\eslang` for Spanish
etc. All these language options are listed in the \ulink[https://petr.olsak.net/optex/]{\OpTeX/ documentation.}

The command `\def\tmark {<mark>}` defines the macro \`\tmark` as a mark
of the translation used (you can remember Translation mark). The marks of all available translations are listed in the file `vars.tex`. 
One of them should be selected as the mark of the currently processed
translation. For example, `BBE` is the mark for the Bible in Basic English.
In the example, six of the usual options for defining a translation marker in the case of English Bibles are given. Only one option (the one actually selected) does not have the `%` comment in front of it. 

 
If you are looking at a book in progress in the `BBE` translation, leave the `main.tex` file in the above form.
When you want to switch to, say, the UKJV, you will use the percent sign to comment out (i.e. make invisible to \TeX) the line with the BBE, but make visible (uncomment)
the line with the UKJV. Then the `\tmark` definition section will look like this:

\begtt
% Translation variants:
%\def\tmark {BBE} %  Bible in Basic English
%\def\tmark {Jubilee2000} % Jubilee 2000
%\def\tmark {NETfree} % New English Tranlation
\def\tmark {UKJV} % Updated King James Version
%\def\tmark {RNKJV} % Restored Names King James Version
%\def\tmark {Webster} % Webster Bible
\endtt

One of the translations must always be active, in other words, the `\tmark` has to be defined. If you forget to put a percent sign before the line you want to comment out, the world won't fall apart; the very last definition that \TeX\ loads, which will redefine any previous ones, will apply.

\`\variants 6 {BKR} {PSP} {CSP} {CEP} {B21} {SNC} 

\hyphenation{Ne-bú-kad-ne-sar}

%Gn 15:13
\vdef {v zemi cizí} 
      {v zemi nepatřící jim} 
      {v zemi, která nebude jejich} 
      {v zemi, která nebude jejich} 
      {v zemi, jež nebude jejich} 
      {v cizí zemi} 

\vdef {v službu je podrobí a trápiti je budou} 
      {budou jim otročit, a utiskovat je budou}
      {budou tam otročit a budou pokořováni} 
      {budou tam otročit a budou tam pokořováni} 
      {budou zotročeni a utiskováni} 
      {budou vykořisťováni a pokořováni} 

%Gn 11:12-22
\vdef {Arfaxad} {Arpachšád} {Arpakšád} {Arpakšád} {Arpakšad} {Arpakšád}
\vdef {Sárug}   {Serúg}     {Serúg}    {Serúg}    {Serug}    {Serúg}   
\vdef {Sále}    {Šálach}    {Šelach}   {Šelach}   {Šelach}   {Šelach}  
\vdef {Heber}   {Éver}      {Heber}    {Heber}    {Heber}    {Heber}   
\vdef {Peleg}   {Peleg}     {Peleg}    {Peleg}    {Peleg}    {Peleg}   
\vdef {Réhu}    {Reú}       {Reú}      {Reú}      {Reú}      {Reú}     
%Ex 6:20
\vdef {Amram} {Amrám} {Amrám} {Amrám} {Amrám} {Amram} 
\vdef {Kahat} {Keháth} {Kehat} {Kehat} {Kehat} {Kehat} 


%Tt 3:9
\vdef {Vyčítání rodů} 
        {Spory a rodokmeny} 
        {Spory, rodokmeny} 
        {Spory o rodokmeny} 
        {Vykládání rodokmenů} 
        {Sporné teologické myšlenky} 
        
\vdef {Putifar} {Pótifar} {Potífar} {Potífar} {Potifar} {Potífarovi} 

%Jb 19:25-27
\vdef   {já vím, že vykupitel můj živ jest, a že v den nejposlednější nad prachem se postaví}  %BKR
        {já vím, že můj Výkupce je živ; on, Poslední, musí povstat nad prach}  %PSP
        {já vím, že můj vykupitel žije a nakonec se postaví nad prachem}  %CSP
        {Já vím, že můj Vykupitel je živ a jako poslední se postaví nad prachem}  %CEP
        {Můj Vykupitel žije, to jedno vím, a že nakonec se nad prachem postaví}  %B21
        {Já vím, že můj Vykupitel žije a nakonec se postaví nad prachem}  %SNC

         
%1Pa 3:15
\vdef   {Jochanan}  %BKR
        {Jóchánán}  %PSP
        {Jóchanan}  %CSP
        {Jóchanan}  %CEP
        {Jochanan}  %B21
        {Jóchanan}  %SNC

\vdef   {Sedechiáš}  %BKR
        {Cidkijjjáhú}  %PSP
        {Sidkijáš}  %CSP
        {Sidkijáš}  %CEP
        {Cidkiáš}  %B21
        {Sidkijáš}  %SNC

\vdef   {Sallum}  %BKR
        {Šallúm}  %PSP
        {Šalúm}  %CSP
        {Šalúm}  %CEP
        {Šalum}  %B21
        {Šalúm}  %SNC

%Ex
\vdef {nebo lid jest tvrdé šíje, a milostiv buď nepravosti naší a hříchu našemu, a měj nás za dědictví}
      {je to lid tvrdé šíje, a kéž odpouštíš stran naší nepravosti a stran našeho hříchu a kéž si nás chceš přivlastnit}
      {je to lid tvrdé šíje. Odpusť naši vinu a náš hřích a přijmi nás do dědictví}
      {Je to lid tvrdošíjný; promiň nám však vinu a hřích a přijmi nás jako dědictví.}
      {je to tvrdošíjný lid. Odpusť naši nepravost i hřích a přijmi nás jako své vlastní}
      {Izraelci jsou tvrdohlavý národ, odpusť nám, prosím, naše nepravosti a provinění a přijmi nás za své}


\vdef {nepravost a přestoupení i hřích}      
      {nepravost a přestoupení a hřích}
      {vinu, přestoupení a hřích}
      {vinu, přestoupení a hřích}
      {nepravost, provinění i hřích}
      {viny, vzdor i nepravosti}

      
\vdef {brány pekelné}
      {brány hádu}
      {brány podsvětí}
      {brány pekel}
      {brány pekel}
      {žádná ďábelská moc}

      

\vdef {Jehoram} {Jórám}     {Jóram}   {Jóram}   {Jehoram} {Jóram}   
\vdef {Joakim}  {Jehójákím} {Jójákím} {Jójakím} {Joakim}  {Jójakím} 
\vdef {Daniel}  {Dánijjél}  {Daniel}  {Daniel}  {Daniel}  {Daniel}   
\vdef {Chananiáš} {Chananjá} {Chananjáš} {Chananjáš} {Chananiáš} {Chananjáš}   
\vdef {Mizael} {Míšáél} {Míšael} {Míšael} {Mišael} {Mišael}    
\vdef {Mizach} {Méšak} {Méšak} {Méšak} {Mešach} {Méšak}   
\vdef {Azariáš} {Azarjá} {Azarjáš} {Azarjáš} {Azariáš} {Azarjáš}   
\vdef {Sidrach} {Šadrach} {Šadrak} {Šadrak} {Šadrach} {Šadrak}   
\vdef {Abdenágo} {Avéd-negó} {Abed-nego} {Abed-nego} {Abednego} {Abed-nego}    
\vdef {Cýr} {Kóreš} {Kýr} {Kýr} {Kýr} {Kýr}   
\vdef {Cýros} {Kóreš} {Kýros} {Kýros} {Kýros} {Kýros}   
\vdef {Cýra} {Kóreše} {Kýra} {Kýra} {Kýra} {Kýra}   
\vdef {Mardocheus} {Mordechaj} {Mordokaj} {Mordokaj} {Mordechaj} {Mordokaj}   
\vdef {Dura} {Dúrá} {Dúra} {Dúra} {Dura} {Dura}   
\vdef {Balsazar} {Bélšaccar} {Belšasar} {Belšasar} {Belšasar} {Belšasar}   
\vdef {Darius} {Dárjáveš} {Dareios} {Darjaveš} {Darjaveš} {Darjaveš}   
\vdef {Izaiáš} {Isajá} {Izajáš} {Izajáš} {Izaiáš} {Izajáš}   
\vdef {šelem} {zvířat} {šelem} {zvířat} {šelem} {šelem}   
\vdef {šelmy} {zvířata} {šelmy} {zvířata} {šelmy} {šelmy}   
\vdef {pard} {levhart} {levhart} {levhart} {pardál} {levhart}   
\vdef {Ezechiáš} {Jechizkijjáhú} {Chizkijáš} {Chizkijáš} {Ezechiáš} {Chizkijáš}   
\vdef {král polední} {král jihu} {král jihu} {král Jihu} {jižní král} {jižní král}   
\vdef {král půlnoční} {král severu} {král severu} {král Severu} {severní král} {král severu}   
\vdef {SINEAR} {ŠINEÁR} {ŠINEÁR} {ŠINEÁR} {BABYLONIE} {Babylonie}    
\vdef {Susan} {Šúšán} {Šúšan} {Šúšan} {Súsy} {Šúšan}   
\vdef {ELAM} {ÉLÁM} {ÉLAM} {ÉLAM} {ELAM} {ÉLAM}    
\vdef {léto} {doba} {čas} {léto} {období} {léto}     
\vdef {svatým výsostí} {svatým Nejvyššího} {svatým Nejvyššího} {svatým Nejvyššího} {ve prospěch svatých Nejvyššího}  {svatým lidem, kteří slouží Nejvyššímu}    
\vdef {Hlava zlatá} {Hlava ze zlata} {Zlatá hlava} {Zlatá hlava} {Zlatá hlava} {Zlatá hlava}   
\vdef {Prsy a ramena} {Hruď a paže} {Hruď a ruce} {Hruď a paže} {Hruď a paže} {Hruď a ruce}   
\vdef {z stříbra} {ze stříbra} {ze stříbra} {ze stříbra} {ze stříbra} {ze stříbra}   
\vdef {Břicho i bedra} {Břicho a bedra} {Břicho a boky} {Břicho a boky} {Břicho a boky} {Břicho a boky}   
\vdef {z mědi} {z mosazi} {bronzové} {z mědi} {bronzové} {z mědi}   
\vdef {Hnátové} {Dolní končetiny} {lýtka} {stehna} {stehna} {stehna}   
\vdef {z železa} {ze železa} {železná} {ze železa} {železná} {ze železa}   
\vdef {Nohy z částky z železa} {Nohy zčásti ze železa} {chodidla zčásti železná,} {Nohy dílem ze železa} {nohy zčásti železné} {nohy někde ze železa}   
\vdef {a z částky z hliny} {a zčásti z hlíny} {zčásti keramická} {a dílem z hlíny} {a z části hliněné} {a někde z hlíny}   
\vdef {Nedvěd} {Medvěd} {"} {"} {"} {"}   
\vdef {Pard} {Levhart} {"} {"} {Pardál} {Levhart}   
\vdef {Hrozná} {Děsivé} {Strašlivá} {Strašné} {Děsivá} {Hrůzostrašná}   
\vdef {šelma} {zvíře} {šelma} {zvíře} {šelma} {šelma}   
\vdef {Skopec} {Beran} {"} {"} {"} {"}   
\vdef {skopce} {berana} {"} {"} {"} {"}   
\vdef {Babylon} {Bável} {Babylon} {Babylón} {Babylon} {Babylón}   
\vdef {Řecko} {Jáván} {Řecko} {"} {"} {"}   


\vdef  {Nabuchodonozor král Babylonský}     
       {Nevúchadneccar, král Bávelu}
       {babylonský král Nebúkadnesar}
       {Nebúkadnesar, babylónský král}
       {babylonský král Nabukadnezar}                  
       {babylónský král Nebúkdnesar}                        
  


\vdef  {Koniáš} %BKR
        {Konjáhú} %PSP
        {Konjáš} %CSP
        {Konjáš} %CEP
        {Jekoniáš} %B21
        {Konjáš} %SNC

%1Pa 3:15
 \vdef  {Jekoniáš} %BKR
        {Konjáhú} %PSP
        {Konjáš} %CSP
        {Konjáš} %CEP
        {Jekoniáš} %B21
        {Konjáš} %SNC

 
 
 
 
 %  Mt 1:11 
\vdef   {Joziáš}  %BKR
        {Jósias}  %PSP
        {Joziáš}  %CSP
        {Joziáš}  %CEP
        {Jošiáš}  %B21
        {Jóšijáš}  %SNC


 
\vdef  {Jeremiáš} %BKR
        {Jeremjá} %PSP
        {Jeremjáš} %CSP
        {Jeremjáš} %CEP
        {Jeremiáš} %B21
        {Jeremjáš} %SNC
  
 
 
 \vdef {Ezechiel} %BKR       
       {Ezekiél} %PSP
       {Ezechiel} %CSP
       {Ezechiel} %CEP
       {Ezechiel} %B21
       {Ezechiel} %SNC
  



\vdef {Zorobábel} %BKR
       {Zerubbável} %PSP
       {Zerubábel} %CSP
       {Zerubábel} %CEP
       {Zerubábel} %B21
       {Zerubábel} %SNC
  

\vdef {Ezdráš} %BKR
       {Ezrá} %PSP
       {Ezdráš} %CSP
       {Ezdráš} %CEP
       {Ezdráš} %B21
       {Ezdráš} %SNC
         

\vdef {Nehemiáš} %BKR
       {Nechemjá} %PSP
       {Nehemjáš} %CSP
       {Nehemjáš} %CEP
       {Nehemiáš} %B21
       {Nehemjáš} %SNC
         

\vdef {Michal} %BKR
       {Mícháél} %PSP
       {Michael} %CSP
       {Míkael} %CEP
       {Michael} %B21
       {Michael} %SNC
         





\vdef {Baltazar}   {Béltešaccar}  {Beltšasar}    {Beltšasar}    {Baltazar} {Beltšasar}   
\vdef {Nabuchodonozor} {Nevúchadneccar}  {Nebúkadnesar} {Nebúkadnesar} {Nabukadnezar} {Nebúkadnesar}   
\vdef {sedm let} {sedm dob} {sedm časů} {sedm let} {sedm období} {po dobu sedmi let}  
\vdef {Daria}   {Dárjáveše}  {Dareia}  {Darjaveše}  {Darjaveše}  {Darjaveš}   
\vdef {Dariov}  {Dárjávešov} {Dareiov} {Darjavešov} {Darjavešov} {Darjavešov}   
\vdef {Dariův} {Dárjávešův} {Dareiův} {Darjavešův} {Darjaveš} {Darjaveš}   
\vdef {Dariem} {Dárjávešem} {Dareiem} {Darjavešem} {Darjavešem} {Darjavešem}   

\vdef {učiněn jest králem} 
        {byl uveden v kralování} 
        {byl učiněn králem} 
        {byl králem} 
        {byl ustanoven vládcem} 
        {byl ustanoven}
          
         
\vdef {ujal království}
      {se kralování ujal}
      {se kralování ujal}
      {se ujal království}
      {obdržel království}
      {začal kralovat}
      



\vdef {Asver} {Achašvéróš} {Achašvéróš} {Achašvéroš} {Ahasver} {Achašvéroš}
  
\vdef {Gabriel} {Gavríél} {Gabriel} {Gabriel} {Gabriel} {Gabriel}
  

%\vdef {Susan} {Šúšán} {Šúšan} {Šúšan} {Súsy} {Šúšan} % nominativ, akuzativ, vokativ
  
\vdef {Mesiáš} {Pomazaný} {Pomazaný} {pomazaný} {Mesiáš} {Mesiáš}
  
\vdef {Elam}  {Élám} {Élam}  {Élam} {Elam} {Élam} 
  
\vdef   {Kanán}  %BKR
        {Kenáan}  %PSP
        {Kenaan}  %CSP
        {Kenaan}  %CEP
        {Kanaán}  %B21
        {Kenaan}  %SNC
  
\vdef {Sidon} {Sidón} {Sidón} {Sidón} {Sidon} {Sidón}  

%Paralipomeon
\vdef {Paralipomenon} {"} {"} {"} {Letopisů} {Paralipomenon}  

%Žalm 7:10
\renum Ž 7:1 = CSP 7:2-18
\vdef   {spravedlivý}  %BKR
        {upřímný srdcem}  %PSP
        {přímého srdce}  %CSP
        {spravedlivý}  %CEP
        {spravedlivý}  %B21
        {kdo žije pravdivě}  %SNC
          

%Ez 3:7
\vdef   {zatvrdilého srdce}  %BKR
        {zarputilý srdcem}  %PSP
        {tvrdého srdce}  %CSP
        {zatvrzelého srdce}  %CEP
        {zarputilý}  %B21
        {zatvrzelý}  %SNC
          
   
% Př 16:5
\vdef   {pyšného srdce}  %BKR
        {pyšný srdcem}  %PSP
        {povýšenec}  %CSP
        {domýšlivec}  %CEP
        {nadutec}  %B21
        {nafoukanec}  %SNC
          


\renum Ž 12:2 = PSP 12:1-8
\vdef   {srdce dvojité}  %BKR
        {dvojité srdce}  %PSP
        {dvojaké srdce}  %CSP
        {obojaké srdce}  %CEP
        {srdce falešné}  %B21
        {obojaká ústa}  %SNC
          
        


\renum Ž 51:3 = PSP 51:1-19
        
% Ž 51:19; 12        
\vdef   {skroušené a potřebné}  %BKR
        {zlomené a zdrcené}  %PSP
        {zkroušené a pokořené}  %CSP
        {zkroušené a zdeptané}  %CEP
        {kajícné a sklíčené}  %B21
        {srdce --- oběť pokání}  %SNC
          
        
\vdef   {čisté}  %BKR
        {čisté}  %PSP
        {čisté}  %CSP
        {čisté}  %CEP
        {ryzí}  %B21
        {čisté}  %SNC
          
         

\vdef   {zmužilého srdce}  %BKR
        {svým srdcem statečný}  %PSP
        {nejodvážnější}  %CSP
        {nejsrdnatější}  %CEP
        {nejstatečnější}  %B21
        {nejodvážnější}  %SNC
          
        
\vdef   
    {kníže veliké, kterýž zastává synů lidu tvého}   %BKR
    {veliký kníže, jenž stojí při dětech tvého lidu}   %PSP
    {velký velitel, jenž zastává tvůj lid}   %CSP
    {velký ochránce, a bude stát při synech tvého lidu}   %CEP
    {veliký kníže a tvého lidu ochránce}   %B21
    {vznešený zastánce tvého lidu}   %SNC
        



%NT

\vdef   {Blahoslavení lkající}
        {Blaženi ti, kteří hořekují}
        {Blahoslavení plačící}
        {Blaze těm, kdo pláčou}
        {Blaze plačícím}
        {Šťastni jsou ti, kteří pláčou v zármutku}
          
        % Mt 5:4 < Jb









\vdef {Neopouštějíce společného shromáždění} %BKR
      {nezanechávajíce svého scházení} %PSP
      {Nezanedbávejme své společné shromažďování} %CSP
      {Nezanedbávejte společná shromáždění} %CEP
      {neopouštějme naše společné shromáždění} %B21
      {Neopouštějte svá shromáždění} %SNC


\vdef   {téhodnů sedm}  %BKR
        {sedm sedmic}  %PSP
        {sedm sedmiletí}  %CSP
        {sedm týdnů}  %CEP
        {sedmero týdnů}  %B21
        {sedm ... týdnů}  %SNC

\vdef   {šedesáte dva}  %BKR
        {šedesát a dvě}  %PSP
        {šedesát dva}  %CSP
        {šedesát dva}  %CEP
        {šedesát dva}  %B21
        {šedesát dva}  %SNC
        
\vdef   {v téhodni posledním}  %BKR
        {jednu sedmici}  %PSP
        {jednoho sedmiletí}  %CSP
        {v jednom týdnu}  %CEP
        {v týdnu posledním}  %B21
        {v posledním týdnu}  %SNC


\renum Da 2:28 = SNC 2:27-27
\renum Da 2:38 = SNC 2:37-37
\renum Da 3:12 = SNC 3:13-13
\renum Da 5:31 = CSP 6:1-1
\renum Da 5:31 = CEP 6:1-1
\renum Da 5:31 = B21 6:1-1
\renum Da 5:31 = SNC 5:31-31 
\renum Da 6:1 = CSP 6:2-29
\renum Da 6:1 = CEP 6:2-29
\renum Da 6:1 = B21 6:2-29
%\renum Da 6:1 = SNC 6:1-28

        

\endinput

\vdef   {}  %BKR
        {}  %PSP
        {}  %CSP
        {}  %CEP
        {}  %B21
        {}  %SNC
        



` reads the configuration about the translation variants
from the `vars.tex` file. 
See section~\ref[vars] for details. Do not touch this line, even though you will probably be editing the `vars.tex` file called by this line.

The \`\input {books.tex}` reads the information about the marks (abbreviations) of the books of the Bible and they are book names are assigned. 
This information is discussed in more detail in section~\ref[books]. 

The macro `\txsfile` (defined by `\def`) specifies the location of `.txs`
files in the directory structure. For each book of the Bible, there has to be
one `.txs` file containing the core text for that book. File
 names vary by book mark, and if there are multiple translations, the file name 
also includes the translation mark. In the `\txsfile` macro, you can use 
 \`\tmark` as the translation mark and \`\amark` or \`\bmark` as the book mark. 
 For book marks, see section~\ref[books], for the format of `.txs` files, see the  discussion in the section~\ref[txs]. 
 In the example,  `.txs` files are  located in the `sources/` directory and are  named `Eng<translation-tag>-<translation-tag-book>.txs`, so for example `EngBBE-Gen.tex`.
 %%To asi potřebujeme doladit. Budeme používat předponu Eng? A pokud ano, kde ano a kde ne?


The English translations mentioned above are ready to go, you don't have to create them for yourself.
If you need some other existing translation, you need to get it into a format usable for OpBible,  in the same form as the `*.txs` files in the `sources/` directory.
The `maketxs` script (see~\ref[txs]) will help you prepare individual `.txs` books from an existing source.

If you happen to be creating a brand new translation and plan to use it with OpBible, it probably wouldn't hurt to compose files one by one for each book directly in the desired format, see also~\ref[txs].

The \`\fmtfile` macro defines the location of the files specifying the formatting
of the core text. Each book of the Bible of each translation used has its own formatting file.
This is something that (unlike the notes) cannot be common to all translations,
because the paragraph breaks and added headings can (and do) differ for each translation.
Our intent was so-called non-destructive editing, in other words, formatting the biblical text
without interfering with it.
These files are discussed in section~\ref[fmt].

The macro \`\notesfile` defines the location of the notes files.
That's  where you will write your commenting notes. 

Each book Bible has its own notes file. 
The notes refer to a place in the core text, and it is the job of \TeX/ to create pages with both the core text and the commenting notes. 
For more details on how to write notes files, see section~\ref[note]. 
Notice that the note files are common to all translations, i.e. there are no separate files distinguished by \`\tmark`. 
The note writing rules allow for the possibility to have various translations over 
one common notation, as long as they are in one language (e.g. English; see section~\ref[translations]). If you want to write
notes for a completely different language, it is the best  to start a new project (preferably in 
another directory) with different `.txs` files, different formatting and note files.

The macro \`\introfile` specifies the files where the introductions to
each book. It is possible (and advisable) to create a separate introduction file for each book.


The macro \`\articlefile` specifies the names of files  which contain the theological articles.
The articles and similar stuff can be placed practically anywhere Bible, on any page with the core text
(i.e., NOT in the Introduction). %\TODO: write...

The macro \`\printedbooks` contains the marks of the books you want  \TeX/ to process.
The sample calls to process the entire Bible, i.e. all 66~books of the Protestant canon.
If you're only doing test prints, for example, you can process only some of the
books of the Bible and have an alternative definition in the main file, for example
`\def\printedbooks{Dan}`. Just put it after the definition for the entire Bible, because
later definitions of the same macro take precedence over any earlier ones.


The \`\processbooks` command starts processing all the books specified in the macro
\`\printedbooks`.  For each book, \TeX/ will read the corresponding core text from
`.txs` file, formats it using the data from the appropriate formatting
file and appends the notes from the appropriate note file.
You don't need to change anything here.

The \`\bye` command will terminate \TeX/.
Anything you type after this farewell to \TeX will be ignored.




You can also add your own macros and settings to the main file before \`\processbooks`, which will affect the entire typesetting completely.

For example, you can put a \`\ChapterPre``{<code>}` or \`\ChapterPost``{<code>}` declaration. These codes are then executed before and after each chapter.
%For example, \readplan -- we need to finish


\secc[vars] File declaring translation variants

If we are working with a single translation variant, there is no need to create this file
and use it. Then just remove (comment out) the instruction to read it from the main file.

In the example in section~\ref[main], the file `vars.tex` is read, which
should contain the declaration of translation variant tags using \`\variants`:
\begtt
\variants <number-of-variants> {<mark>} {<mark>} ... {<mark>}
\endtt
where `<number-of-variants>` is the number of translation variants (into a common language,
for example, English), and then all the tags of the variant translations are listed.
For example,
\begtt
\variants 6 {BBE} {Jubilee2000} {NETfree} {UKJV} {RNKJV} {Webster}
\endtt
declares the abbreviations for the 6 English translation variants:
BBE for  Bible in Basic English, Jubilee2000 for Jubilee 2000, NETfree for New English Tranlation, UKJV for Updated King James Version, RNKJV for Restored Names King James Version, and Webster for the Webster Bible.


The translation variants thus defined must match the definitions of `\tmark` in the main `main.tex` file, including upper or lower case.

Consider in advance the number of translations you want to use (changing their number later will be very difficult, though not impossible)
and especially the order of the translations: the same order in which they are declared in the definition of `\variants`  will apply to the entire project.
In all the notes commenting on a phrase that spells differently in different translations, you will list the different versions in that precise order.


If you know that a phrase or word will appear more often than just in a single comment, 
you can define it directly in this `vars.tex` file using the \`\vdef` command.
The number of phrases listed after \`\vdef` must be exactly the same as the number of
`<number-variant>`, each of them enclosed in brackets, and they must correspond to the translation variants in the same order as the variants listed in the declaration of \`\variants`. For example:





\begtt
\vdef {Greece} {Greece} {Greece} {Grecia} {Grecia} {Grecia}
\endtt
declares that the name of Greece is transcribed differently in different
variants of translation: it is {\it Greece\/} in BBE, Jubilee2000, and NETfree, but changes to Grecia in UKJV, RNKJV, and Webster. 

When we write notes concerning this country in the
notes file, we will just write `\x/Greece/` (the first translation variant in the definition of `\variants`)
and this will be turned into the corresponding phrase used in the currently processed
translation that we have declared in the main file using `\def\tmark{...}`.

So after changing `\def\tmark{...}` in the main file, all occurrences of
of `\x/Greece/` in the text of notes will automatically start behaving differently
and adapt to the phraseology of that particular translation variant.
Then such words can be inflected or added various endings, for example: The entry `\x/Greece/'s` will yield the form `Grecia's` in the note under the UKJV, RNKJV, and Webster translations but will remain `Greece's` with BBE, Jubilee2000, and NETfree. 
This is discussed in more detail in section~\ref[translations].

The command \`\variants` declaring the abbreviations of the translation variants is unique (the only one) for the variants file, 
whereas there can be more \`\vdef` commands defining variant phrases in the file, because there are of course many phrases that are used in different
translation variants,  not just the country of Greece.


The whole passages of text can be handled differently depending on the translation variant set. The branching command  \`\switch`is used for this purpose. 
It is discussed in more detail in the section~\ref[switch]. 
For example, the names of individual translations (which are then used in the page header) can be declared differently for different translations using `\def\bibname`:
\begtt

\switch {BBE}{\def\bibname{Bible in Basic English}}%
        {Jubilee2000}{\def\bibname{Jubilee 2000}}%
        {NETfree}{\def\bibname{New English Tranlation}}%
        {UKJV}{\def\bibname{Updated King James Version}}%
        {RNKJV}{\def\bibname{Restored Names King James Version}}%
        {Webster}{\def\bibname{Webster Bible}}%

\endtt
This particular declaration is a part of the already prepared `vars.tex` file.

\secc[books] Books names file

In the main file, there is an instruction to read the books names file, for example
`\input {books.tex}`. That file must contain the commands
\`\BookTitle` in the format:
\begtt
\BookTitle <a-mark> <b-mark> {<non-abbreviated book title>}
\endtt
There must be at least one space between the marks and the book title.
The beginning of a file read this way might look like this:

\begtt
\BookTitle Gn Gen {The First Book of Moses (Genesis)}
\BookTitle Ex Exod {The Second Book of Moses (Exodus)}
\BookTitle Lev Lev {The Third Book of Moses (Levicitus)}
\BookTitle Nu Num {The Fourth Book of Moses (Numeri)}
\BookTitle Dt Deut {The Fifth Book of Moses (Deuteronomy)}
\BookTitle Jos Josh {Joshua}
\BookTitle Jdg Judg {Judges}
...
\endtt
In the first column after \`\BookTitle`, there are `<a-marks>`, which are further
used in the text of the notes and are used to create links to the Bible passages and other notes. 


In the second column, there are `<b-marks>`, which can be the same as
`<a-marks>`, but may also be different. It is possible, for example, that the names of the `.txs`
files were created by exporting from some software and the bookmarks are different than
we need to use in the text of our notes. Then it is possible that in the main file
declare the location of the `.txs` files using \`\bmark` instead of \`\amark`, i.e.
\begtt
\def\txsfile {sources/Cze\tmark-\bmark.txs}
\endtt
and have the files `CzeBKR-Gen.txs`, `CzeBKR-Exod.txs`, while in the text (including the references) we use
the marks `Gn`, `Ex`, etc., not `Gen`, `Exod`.

The macro \`\amark` contains the `<a-mark>` of the currently processed book and the macro
macro \`\bmark` includes the `<b-mark>` of the book currently being processed.

Note that the macro \`\printedbooks` (in the file `main.tex`) with the marks of all the books we want to
to process (see section~\ref[main]) contains `<a-marks>`, not `<b-marks>`.

In the third parameter after \`\BookTitle`, the book names are in brackets.

File with 66 entries of \`\BookTitle` %(Protestant canon) The 
is generated automatically after extracting
the core texts from Sword download using a program `mod2tex` and a python script `maketxs` (see
section~\ref[txs]). Feel free to use them, but `<a-tags>` and book titles will
probably need to be manually modified according to the conventions of the translation, as demonstrated
in the example above.

Additional information about individual books can be added to the book title file
using the \`\BookException`, \`\BookPre`, \`\BookPost` commands. They have the following syntax:
\begtt
\BookException <a-mark> {<exception-text>}
\BookPre <a-mark> {<text-before-book>}
\BookPost <a-mark> {<text-after-book>}
\endtt

The <exception-text> is inserted before reading all files of the book defined by the <a-mark>
inside the cycle  for reading all books with `\processbooks`. Then
the <text-before-book> is inserted after the book files have been read but before
the first verse is processed. Finally, the <text-after-book> is inserted after the last
verse of the book.

Example of using \`\BookException`:
Let's assume your language has diacritical marks that happen to be a part of the abbreviations of some Biblical books, and even though you cannot use diacritics in the file names, you still want to use them in your text  to refer to those books as they appear in your Bible.  
What you can do is to modify the value of the macro \`\amark` so that it does not contain
diacritics as follows (following example applies for Czech language):
\begtt
\BookException Ž {\def\amark{Z}}
\BookException Př {\def\amark{Pr}}
\BookException Pís {\def\amark{Pis}}
\BookException Ř {\def\amark{R}}
\BookException Žd {\def\amark{Zd}}
\endtt

Then the `notes-\amark.tex` files, for example, are actually named `notes-Z.tex`,
`notes-Pr.tex` etc. In the notes text you normally use the  marks of usual book abbreviations, e.g.,
Ž, Př, Pís, etc.

There are five books in the Bible that have only one chapter (Obadiah, Philemon, 2 and 3 John and Jude).
Because the references to them are not written with the chapter number (`Ph 1:4`) but only with the verse number (`Ph 4`), 
we have to teach \TeX\ which ones they are so that it does not expect the chapter number but knows that it's the number 1 which in turn is not written anywhere.
When you refer to one of such books, the reference is being interpreted in a different way, see section~\ref[vudaj].
This is achieved by defining a macro \`\nochapbooks` which must contain the <a-marks> of these books:
`\def\nochapbooks {Ob Pm 2Jn 3Jn Jd}`, obviously identical to those already listed in the `\BookTitle` definition (in the `books.tex` file).


\secc[txs] Core text format, called `.txs` files

The core text of the Bible is assumed to be stored in `.txs` files.
(text source). Each of the ~66~ books of the Bible is stored in its own `.txs` file.
The names of the `.txs` files and their locations must match the `\def\txsfile` declaration
in the `main.tex` file (see~section~\ref[main]).

Each line of the `.txs` file contains one Bible verse started with 
`#<chapter-number>:<verse-number>`. The verses must be listed in the correct
order. For example, the beginning of the `BBE-Dan.txs` file looks like this (parts of the text are omitted in the sample):

\begtt
#1:1 In the third year of the rule of Jehoiakim, king of Judah ... shutting it in with his forces.
#1:2 And the Lord gave into his hands Jehoiakim, king of Judah ... and he put the vessels into the store-house of his god. 
...
\endtt


The core texts of the Bible can be obtained, for example, from the Sword modules 
\url{https://www.crosswire.org/sword/modules/ModDisp.jsp?modType=Bibles}.
Individual `.txs` files can then be generated using the following procedure (on Linux):
%http://www.biblesupport.com/e-sword-downloads/file/2574-english-jubilee-2000-bible-jb2000bblxexe/ %e-Sword module installer for Windoze
Unzip the ZIP downloaded from the  web page above (the so-called module). You need
have the `libsword-dev` package installed on your computer and the program
`mod2tex`, which is part of \OpBible/. Use `installmgr -l` to find out
list of downloaded modules. If your current directory is the location where you
have unzipped the ZIPs and where the `modules` directory was created, then the modules will be found. The modules contain text in binary format and we need 
to convert them into text format. To do this, just type the following into the command line: 
\begtt 
mod2tex module > file
\endtt 
where `module` is the name of the module. In the resulting file
you have the complete base text of the translation (module).
For example, after
\begtt
mod2tex KJVA > KJVA.out
\endtt
the complete translation of the King James Bible (including Apocrypha) is in the file `KJVA.out`.
This can now be split into `.txs` files with the command
\begtt
maketxs KJVA.out
\endtt
This command will create the `KJVA-books.tex` file in addition to the 66 `.txs` files, in which
which contains the titles and abbreviations of each book, so there is:
\begtt
\BookTitle Gen Gen {Genesis}
\BookTitle Exod Exod {Exodus}
\BookTitle Lev Lev {Leviticus}
\BookTitle Num Num  {Numbers}
\BookTitle Deut Deut {Deuteronomy}
\BookTitle Josh Josh {Joshua}
\BookTitle Judg Judg {Judges}
...
\endtt

These titles are in English, as the Sword source does not use non-English module names. 
It is therefore it is necessary to manually edit this file and insert the titles of your local language instead of English ones. 
The abbreviations of the books are listed twice as well. The first one should be changed according to convention of abbreviations in your language's Bibles and the second column of abbreviations can stay in English as it already is (as these second abbreviations, the <b-marks>, will be part of the `.txs` file names).  

Then the declaration 
\begtt
\def\txsfile {sources/Eng\tmark-\bmark.txs}
\endtt
in the main file will cause the created `.txs` files to be searched for 
in the `sources/` directory and their names are assumed
`EngKJVA-Gen.txs`, `EngKJVA-Exod.txs`, `EngKJVA-Lev.txs`, etc.

If the rare case should arise that you were to compile the Bible from several different sources, say the Old Testament you wanted in Dr. Jan Hejčl's translation and the New Testament in František Žilka's translation, you would have to play a bit with the file names so that the resulting definition of `\tmark` would be the same for the whole Bible. The two `*.out` files (e.g. `HEJCL.out` and `ZILKA.out`)\fnote{Their modules are not on Sword, the `*.out` files would have to be created by downloading them from e.g. `https://obohu.cz` and converting them to the desired form with some clever script (or manually?).} would have to be merged into one and then named e.g. `CzeHecjlZilka.out` and  only then execute `maketxs CzeHejclZilka.out`.  Now you can have a `\def\tmark{HejclZilka}` definition in the main `main.tex` file and 
the resulting study Bible will have Hejčl's Old Testament and Žilka's New Tenstament.




From now on, you no longer need to modify  `.txs` files in any way.
In the `sources/` directory, you can have a “data store of all the core Bible texts for all used translation variants at one place. In the case of the six variants
translation of Protestant canon, you have $6 times 66=396$ files.



If you have something in the `.txs` files that you want to format differently, it is 
possible to use \`\cnvtext``{<what>}{<how>}` in the main file. 
\TeX/  will look up all occurrences of `<what>` in each verse of `.txs` file and replace them with `<how>`.
For example, if you have sections of text in square brackets in the `.txs` files, i.e.
`[something like this]` and you want to print them in italics, write in the main file:
\begtt
\cnvtext{[}{\bgroup\it} \cnvtext{]}{\/\egroup}
\endtt

It may turn out that the `.txs` file does not use the correct typographical quotation marks (i.e. in English “...”), but instead there are the programmer's quote marks {\tt"}...{\tt"}. 
Without interfering with the `.txs` file, this can be fixed by adding an instruction to
the main file:
\begtt
\quotationmarks{“}{”}
\endtt
This will automatically replace the programmer's quote marks in the `.txs` file with correct English  typographical quotation marks. 
Similarly, you can declare the replacement of the programmer's quote marks with 
with Czech or any other other quotation marks, for example by declaring
`\quotationmarks{`„`}{`“`}`.
The programmer's quote mark is then implicitly
replaced by the opening typographical quotation mark (the first parameter in the 
declaration), but if it is followed by a space, end-of-verse, end-of-paragraph, period, 
or comma, it is is replaced by a closing typographic quotation mark (the second 
parameter).

\secc[fmt] Entries specifying the formatting of the base text in `fmt-*.tex` files

The core text in `.txs` files does not contain any formatting or additional information, 
such as chapter titles or where to end a paragraph or where to switch from block format 
to center-justification   and back.

%\mnote{\code{\\fmtadd} \code{\fmtpre} \code{\fmtins}}

Since we don't want to interfere with the core text, \fnote{Think of them as the
Holy Scriptures, therefore an “untouchable” texts. The only reason that might entitle 
you to interfere with the basic text is the unlikely possibility that you might find an 
error in the source from the Sword (or wherever you've got it from), when compared against the hard copy. Then it is really better to fix such error on the spot.} 
you need to declare this 
additional information and link it to the corresponding verses using the special 
commands \`\fmtadd`, \`\fmtpre` and \`\fmtins`. 
These commands are typically in the `fmt-*-*.tex` files,
for example, `fmt-NETfree-Dan.tex`. It is advisable to maintain these formatting files
dependent both on the book (Daniel in the example above) and on the
translation (the King James Version with Apocrypha in the example). 
You can always  start with one file for each book and  create the files for  other translations as copies of this default one,  at the end of a day, however,
it may turn out to be be necessary to modify the formatting instructions
for different variants of the core text according to the translation used.


The syntax for using these commands is as follows:
\begtt
\fmtpre{<chapter-number>:<verse-number>}{<fmt-command>}
\fmtadd{<chapter-number>:<verse-number>}{<fmt-command>}
\fmtins{<chapter-number>:<verse-number>}{<phrase>}{<fmt-command>}
\endtt
where <fmt-command> is the ``command'' given to \TeX/ for formatting.
%\mnote{\code{\\endgraf} \code{\\begcenter} \code{\\endcenter}}
For example, \`\endgraf` marks the end of a paragraph; \`\begcenter` opens a passage with
centred text which must be somewhere later closed  with \`\endcenter`. Or 
\`\chaptit{<text>}` inserts <text> as the chapter title, whereas
\`\schaptit{<text>}` inserts the pericope title elsewhere, other than before the first verse of the chapter, and makes space above and below that title.

%\mnote{\code{\\fmtadd} \code{\fmtpre} \code{\fmtins}}
The \`\fmtpre`  inserts <fmt-command>
at the beginning of the specified verse (before the verse number which is printed in a form 
of a top index). 
The \`\fmtadd`  inserts <fmt-command> at the end of the specified verse.
Finally, \`\fmtins` inserts <fmt-command> inside the verse after the first occurrence of the 
specified <phrase> that must literally exist in the verse. Otherwise, \TeX/ prints a warning 
message and inserts no <fmt-command> whatsoever.

An example of how the `\fmt*` commands can be used can be seen in the file
`fmt-BBE-Dan.tex`.

The command \`\fmtfont``{<chapter-number>:<verse-number>}{<phrase>}{<font>}`
is used to highlight the selected phrase with the selected font.
For example, the `\fmtfont{1:26}{people}{\em}` in the book of Genesis will print the word
\"people"  in italics, because `\em` is an intelligent italics switch (it automatically adds the italics correction `\/` after the word, which you don't have to worry about, but which you would if you switched to italics with the regular `\it`).
Any other font switch can be used instead of `\em`.

In addition to `\begcenter` and `\endcenter`, it is also possible to use controlled 
indentation with \`\ind``<number>` (as indent). 
At the point of insertion, the line is terminated and the next line begins indented by the
`<number>` of paragraph indents. However, inserting such `\ind` commands
via `\fmtins` or `\fmtpre` can be quite laborious and
cluttered, yet the Bible is rife with poetic passages that require a lot of
of differently indented lines. The command \`\fmtpoetry` can be used for this purpose.
We will first demonstrate its use in the example of the NETfree translation of 
Daniel 6:26--27:

\medskip
\picw=.5\hsize
\centerline{\inspic{images/NETfreeDan-crop.pdf} }
\vskip-2pt % it's on two sides, so it looks a little smooth



The poetic part of this sample was typeset this way:
\begtt
\fmtpoetry{6:26}{God; // forever. / destroyed; //}
\fmtpoetry{6:27}{/ delivers // wonders /// earth. / Daniel // }
  \fmtins{6:26}{Daniel.}{\medskip\hglue-2mm}
  \fmtadd{6:27}{\bigskip}
\endtt








\`\fmtpoetry``{<chapter-number>:<version-number>}{<formatting-data>}` determines the 
format of a particular verse. The <formatting-data>
contain words from the ends of a line followed by one or more slashes. Number of
of slashes indicates how many paragraph indents the next line after the
that word is shifted to the right. <formatting-data> must necessarily end with one or more
slashes and may (but do not have to) begin with one or more slashes.  If
the do begin, the beginning of the line is indented by the appropriate number of paragraph indents,
but the verse number is set off slightly to the left in the space of a paragraph
indentation.

If we want to insert extra vertical spaces when using \`\fmtpoetry`, it is
is possible, but only after the `\fmtpoetry` command, as you can  see in the example where
\`\medskip` (half line space) is inserted. 
The rule is that when you use more formatting commands like `\fmtins` or `\fmtpre` in poetry, 
these multiple formatting instructions will eventually be executed in
in the reverse order, so for example, for the verse 6:26
in the above example, first `\medskip` is executed at the beginning of the poetic part of the verse 26, after the name Daniel followed by a period: `\fmtins{6:26}{Daniel.}{\medskip\hglue-2mm}`
Then indent of the next line is reduced by `\hglue2mm` in order to stick the quotation marks out of the poetry block. 
% To zrušíme, jestli půjde uvozovky vystrkovat automaticky. Kdyby ne, takhle to může fungovat taky.



\secc Notes in `notes-*.tex` files

These files (located according to the `\notesfile` declaration in the main file)
contain, among other things, notes on individual verses or parts of verses. 
 The command \`\Note` will be the most frequent command in the `notes-*.tex` files, designed to type in the particular notes.  Since this commentary apparatus is the main reason why the
the \OpBible/ package was created, an entire section~\ref[note] is devoted to it.

\sec Notes and other objects linked to the core text

\secc[note] Notes linked to core text phrases: The `\Note` command

The main purpose of the \OpBible/ tool is to create a PDF from the core text not only
with the Bible text itself, but with notes linked to their verses. 
In order to achieve that we use the `notes-*.tex` files (for example, `notes-Gen.tex` for the book of Genesis), which contain notes on individual phrases of the core text according to
the following  format. Each individual note is prefixed with the command \`\Note` in this required form:




\begtt
\Note <chapter-number>:<verse-number> {<phrase>} <note-text>
<blank line>
\endtt
For example:
\begtt
\Note 1:2 {the temple of his god} 
Marduk was the chief god of the Babylonian pantheon (cf. <Jer 50:2>).
\endtt
The example is from the `notes-Dan.tex` file, i.e., the notes file on the book of Daniel. 
Specifically, the note refers to chapter one, verse two, and the phrase “the temple of his god.”
This phrase, case-sensitive, must necessarily exist in the specified verse
of the core text. Then \TeX/ links it to the corresponding place 
of the core text,  ensuring that the phrase and its commenting note occur
on the same page. In other words, the pages do not break by verse number and its corresponding note number; they break by the commented phrase.
When a page break occurs within a verse in which more than one phrase is being commented on, the notes follow their phrases, not just the beginning of the verse.
See an example of this in section~\ref[break-verse].





If the <phrase> is not present exactly in the specified verse of the core text,
\TeX/ reports a warning to the log and to the terminal during processing and
matches the note with the given verse as if the <phrase> were at the beginning of the verse.

The printed note (depending on the typographic design) contains
a repetition of the chapter and verse numbers, followed by the <phrase> that is being commented on
(for example, in bold) followed by the actual text of the note.

Sometimes it is necessary to search for a slightly different phrase in the core text than what
we want to have in the printed note (e.g., the word in the core text is in
a different case, or it is a slightly differently worded phrase). Then it is possible to type an equal sign immediately after the `{<phrase>}`, which in turn is followed by the  `{<phrase-to-be-printed>}`, i.e.,

\begtt
\Note <chapter-number>:<verse-number> {<phrase>}={<phrase-to-be-printed>} <note-text>
<blank line>
\endtt
In this case, <phrase> is searched for in the core text, but in the actual
note, <phrase-to-be-printed> is printed as the note entry. For example:

\begtt
\Note 2:32-33 {head}={The head ... gold, its chest and arms ... silver, its belly and thighs ... bronze, its legs ... iron, its feet partly ... iron and partly ... clay} 
Moving from the head to the feet of the image, there is a decrease not only in the  weight of the materials but also in its value. The image was clearly too heavy with fragile feet. It is an illustration of the fate of all human kingdoms and civilizations:  at the end, each of them will collapse by its own weight.
\endtt

In this example, the phrase “head” is being searched for but the text “The head ... gold, its chest and arms ... silver, its belly and thighs ... 
bronze, its legs ... iron, its feet partly ... iron and partly ... clay” will be printed in bold font as the first part of the note.
In this particular example, the note (or at least its beginning) will occur on the page where the first occurrence of the word “head” in the verse 2:32 takes place.

It is  possible to have more notes on the same Bible verse. Each of the notes, however, have to be separated by a blank line as always. 
In case that two or more consequent notes have identical <chapter-number>:<verse-number>, these numbers are printed only in the first instance, not in following notes on the same verse.

The individual notes in the source notes file are separated by blank
lines. This is necessary, otherwise \TeX/ would not know where the text ends when reading them. 
It also increases the clarity of the source file. 
%Additional lines of the note can (but need not) be indented. %really needed?

If the note refers to the whole verse (without a specified phrase), write
`{}`, i.e., an empty search phrase. For example:
\begtt
\Note 1:1-21 {}={Keeping ritual purity} The prophet introduces the context of his book by recounting a 
personal history (of his and of his friends) in the captivity, education, loyalty to God and service 
to King Nebuchadnezzar.
\endtt

Moreover, the last two examples also demonstrate the possibility to give a range of verse numbers.
The verse range will be printed in the beginning of the note as expected.
If the <phrase> for the search is blank (as in this very last example), then the note
will be placed on the same page as the beginning of the first verse in the~specified range.
If <phrase> is non-empty, it must occur in the first verse of the range of verses. 
The range symbol “`-`” is the only “minus” character normally available on
keyboard. Its code is~45. It must not be any special character that somehow resembles a horizontal 
dash.







The order of printed notes on the same verse corresponds to the order of the phrases on which
they comment in the core text. That means that the order in which they are written 
in the source file is irrelevant. Notes that link to an entire verse using the empty parameter `{}`  are placed as the first, and if there are more than one of them, only then their order follows that in the source file.




\secc[break-verse] Page break within a verse  

If different notes comment on different phrases within a single verse, and at the same time a page break after all formatting happens to come between those phrases in the core text, then the notes follow their phrases to their respective pages. In other words, the page break is not determined by the verse number, but by the phrase found  in the core text and synchronized with the note. 

In the sample Book of Daniel (of the Webster Bible translation), you can see this happen at the verse 10:20 
(all the searched phrases are highlighted in red color). Look at page 23
where you read “will I return to fight with the prince of Persia” and then at page 24 where the same verse continues with the phrase “the prince of Grecia.” The commenting notes follow their respective phrases on correct pages. 
(You probably do understand that in a real hard copy Bible the page 23 would be the right one, as all odd pages are, and 24 would be its other side after you turn the page, don't you?)

\bigskip
\picw=200pt
\line{\hskip10pt\inspic{images/Webster-Dan-10-20A-crop.pdf} \hss
                        \inspic{images/Webster-Dan-10-20B-crop.pdf} 
                       \hskip10pt}


\secc[articles] Commands to insert other objects

In addition to the \`\Note` commands, the `notes-*.tex` file can be used to write instructions for inserting additional objects related to a particular verse in the core text. Such an object is placed on the bottom of the page under the two columns of notes.\fnote{According to the implicit definition of the page design; this can be redefined.} If the object is a quotation, it is placed on
top of the page. We define the link to the location of the text using the 
`<chapter-number>:<verse-number>`, similar to \`\Note`. The object is placed on the same page where the corresponding verse begins.
If the object does not fit on the page below the verse, it is placed at the bottom of the
of the following page.
%% I guess it would take some examples

You can insert images using \`\putImage` (section~\ref[putimage]), articles using \`\putArticle` (section~\ref[putarticle]), quotations using \`\putCite` (section~\ref[topcite]), images across two pages in an open book using \`\putSpanImage` or \`\putSpanText` (section~\ref[spanimage]).

The order in which you write the commands to insert these objects in the `notes-*.tex` file has no effect on the final appearance. You can have all objects of one type concentrated in one place in the source file, or you can  have them placed between `\Notes`, typically by the verse number to which the objects are bound.





\secc[putimage] Inserting Images

To place images with a link to a specific verse, use the \`\putImage` command:

%\mnote{\code{\\putImage}}
\begtt
\putImage <chapter-number>:<verse-number> {<caption>} [<label>] (<parameters>) {<file>}
\endtt
For example:

\begtt
\putImage 2:1 {Daniel's Visions} [daniels-visions] () {Nabuco.pdf}
\endtt



\begitems
* `<chapter-number>:<verse-number>` indicates the location where the image is placed.
* `<caption>` is the caption to the image. 
In the default typographic settings, it is printed in a strip that separates the upper part of the page from the  the image itself.
* `<label>` specifies the image's  reference label. You can then refer to it using \hfil\break
\code{...see image \\ref [<daniels-visions>] on page \\pg}, which, 
%  ...see image` \`\ref` `[<daniels-visions>]` on page \`\pg`, which,  
  in the case of our example, prints ... see image {\Blue Daniel's Visions} on page~123.
  In doing so, the text in this link will be hyperlinked to the place where the image occurs. 
  If you don't want to use `<label>`, type `[]`   (an empty label).
  
* `<parameter>` specifies optional additional instructions for formatting the image. They can  be non-present, as the example suggests.
  By default, the image is stretched to the full width of the page. If you want it
  smaller, for example, type `<parameters>` in the space for `\picw=9cm` which
  will cause the image to be 9\,cm wide and centered.
* `<file>` is the full name of the image file. It can have the extension `pdf` (for
  vector images) or `png` or `jpg` (for bitmap images).
  Image files must be stored in the `images/` directory. If they
  are located elsewhere, the parameter \`\picdir` must be set for that location,
  for example `\picdir={BibleImages/}`. It is sensible to have it set in the main file, see section~\ref[main].
\enditems



\secc[putarticle] Inserting Articles

Articles with text (typically significantly longer than a note) can be placed on a page in a similar matter as
images, i.e., in the default typographic setting at the bottom of the of the page on which the verse specified by
`<chapter-number>:<verse-number>` happens to occur. 
If it does not fit there, it is inserted at the bottom of the  next page. 
If it does not fit on a single page, its next part is inserted  at the bottom of the next page (and so on until the whole article is complete).



Write the text of  articles for each Biblical book in a file called by the macro \`\articlefile`. 
For example, the file `articles-Gen.tex` contains all the articles for the book of Genesis. 
For how the content of such file has to look like, see below.


In the `notes-*.tex` file, you need to specify the article location request
using \`\putArticle`:


\begtt
\putArticle <chapter-number>:<verse-number> {<title>} [<article-number>] (<parameters>)
\endtt
For example:
\begtt
\putArticle 6:1 {Who Was Darius the Mede?} [6] ()
\endtt

\begitems
* `<chapter-number>:<verse-number>` indicates the location to which the article will be placed.
* `<title>` is the title of the article. It is printed in a similar way as the caption of an image.
* `<article-number>` is a numeric designation of the article that must be unique within each single Biblical book. 
As articles are typically linked to chapters (we don't assume more than one article in a single chapter), then the
  `<article-number>` can (and should but does not necessarily have to) be the number of the corresponding chapter. 
  The article is not placed following this number, however, the nubmer is being used for references.
 For example, you can type \code{...see <"article" Da 6>a} and it will print
  ...see {\Blue article~Da~6}. The format and properties of such links
  are described in section~\ref[links]. In addition, the `<article-number>` is used to find
  the actual text of the article in the corresponding `atricles-*.tex` file.
The * `<parameters>` are optional parameters that specify the formatting of the article. 
If you are satisfied with the default settings (e.g., the width = `\hsize`) then leave the parentheses `()` empty.
  \TODO...
\enditems

The text of the articles to be inserted must be in the `articles-*.tex` file.
This file must contain a line started by \`\Article` as follows:

\begtt
\Article [<article-number>]
\endtt
This is followed by the text of the article. Then it can continue with the next
`\Article` `[<article-number>]` followed by more text, etc. In one file
all the articles for a given book are thus accumulated.

If you specify the request \`\putArticle` but the corresponding file `article-*.tex`
or the corresponding line  \`\Article` `[<article-number>]` does not exist, the \TeX will end with an error.





\secc[topcite] Inserting quotations at the top of the page

%\mnote{\code{\\putCite}}
The command \`\putCite` `<chapter-number>:<chapter-number> {<text>}`, 
in default typographic setting, inserts <text> as a quotation at the beginning of
page containing `<chapter-number>:<verse-number>`.

The <text> itself may contain a \`\quotedby {<author>}` at the end. In such a case
 <author> is printed on a new line (if the  quotation is placed on the right page) or, assuming there is enough space, it is shifted more to the right on the last line (if the quotations  is placed on the left page).



For a controlled transition to a new line, you can use the \`\nl` (new line) command in <text>.


\secc[cite] Inserting qutations in the margin of an article 

Article text is formatted in two columns by default.
It is possible to break the outer column and insert a quotation or something similar.
This text will stick out into the outer margin.

Inside the article (i.e. just after \`\Article` `[<article-number>]`) 
you need to insert a declaration of the quoted text using:
\`\Cite` `<letter> {<text>}`. %\mnote{\code{\\Cite}}
Here <letter> is typically `A`. But if you want to insert more than one
quotations in a single article, you need to distinguish them with additional letters, i.e. `B`, `C`, etc., and all quotations have to be written at the beginning of the article.


%\mnote{\code{\\insertCite}}
The \`\Cite` command only declares the quote. Its actual insertion into the text
is done by \`\insertCite` `<letter>\left` and at the same time  \`\insertCite` `<letter>\right` elsewhere in the text of the article.
If the article is on the left-hand side of a double page, the quotation is placed
only by `\insertCite<letter>\left`, in other words `\insertCite<letter>\right` is
ignored. If the article is on the right-hand side of an open double page, the placement is
is governed only by `\insertCite <letter>\right`. The quotation will appear
right below the line in which the `\insertCite` is given. The line itself is
is not split because of this (in other words, the paragraph is not terminated because of the insertion of the quotation).

The reason why it is necessary to give two locations for \`\insertCite` is  following: 
We don't know ahead on which page (odd or even) the article with the inserted quotation
will appear. Since the quote should stick out in the outer margin, it should be placed on the left
page in the first column, and on the right page in the second column, that is 
somewhere slightly different. The location of `\insertCite <letter>\left` should therefore
correspond to a line in the first column and `\insertCite <letter>\right`
to another line in the second column of the article.


It is wise to debug (i.e., test in advance) what the placement of the quote looks like
for both options that may occur (left/right). If you want to
to see how it works for the variant that doesn't match the correct position
of a page, you can use the command \`\swapCites` at the beginning of the article text which will 
switch the left/right position of the quotation.
%\mnote{\code{\\swapCites}}
But this should not be left on for the final printing, thus activated
`\swapCites` will cause a warning to the terminal and to the log file.

\sec[translations] Different (but similar) versions of the core text

See section~\ref[vars] for an introduction to this issue. This is detailed documentation on each 
option.

\secc[cmdx] Declaring translation variants and using the `\x` command

If the variants are not declared with the \`\vdef` command, then
\`\x``/<phrase>/` command used in the text will output <phrase>. However, it is possible to
declare translation variants. The number of variants must be specified with the `\variants` command.
(see section~\ref[vars]). This is done once in the whole document. Then the `\vdef` commands can follow, always with as many parameters as the number of variants specified by the `\variants` command. For example, 

\begtt \typosize[8.5/11]
\variants 6 {BBE} {Jubilee2000} {NETfree} {UKJV} {RNKJV} {Webster} 

\vdef {took the kingdom} % BBE
      {possessed the kingdom} % Jubilee2000
      {take possession of the kingdom} % NETfree
      {possessed the kingdom} % UKJV
      {possessed the kingdom} % RNKJV    
      {possessed the kingdom} % Webster
      
\vdef {He-goat} % BBE
      {Goat} % Jubilee2000
      {Male goat} % NETfree
      {He goat} % UKJV
      {Goat} % RNKJV
      {He-goat} % Webster

\endtt

If `\def\tmark {<variant>}` is now declared in the main file,
then the `\x``/<phrase>/` will turn into the <phrase> of the specified <variant>.
In doing so, the <phrase> parameter of the \`\x` command must be identical to the~first phrase specified in the \`\vdef` command. If, in our case, the main file says `\def\tmark{BBE}`, then
\medskip
\code{\\x/took the kingdom/} yields: took the kingdom 

\code{\\x/He-goat/} gives: He-goat

\code{\\x/Whatever/} prints: Whatever

\medskip

But if there is a `\def\tmark{Jubilee2000}` in the main file, then

\medskip
\code{\\x/took the kingdom/} yields: possessed the kingdom 

\code{\\x/He-goat/} gives: Goat

\code{\\x/Whatever/}  will print Whatever and the terminal will warn
                                  that the phrase `/Whatever/` has no declared translation.
\medskip




%For example, a note in the `notes-Da.tex` file (for the book of Daniel) might read:
%\begtt \catcode`\<=12
%\Note 5:31 {Darius}={\x/Darius/Median} Some schools claim that this and other 
%   (<6:1>, <6:6>, <6:9>, <6:25>, <6:28>; <9:1>; <11:1>)     
%   The references to \x/Darius/ Medes in the book of \x/Daniel/ are historical errors.
%\endtt
%and this remark is printed at `\def\tmark{NETfree}` as:
%
%{\medskip\leftskip=\parindent\noindent
%{\bf 5:31 Darius the Mede} \ Some schools claim that this and other 
%(6:1, 6:6, 6:9, 6:25, 6:28; 9:1; 11:1)     
%references to Darius the Mede in the book of Daniel are historical errors.
%\medskip}

Selected parameters of \`\vdef` may be empty (written as ~`{}`),
indicating an undefined phrase for that language.
If such a phrase needs to be used with \`\x/.../`, a warning will be printed.
In addition, the parameter may contain a single `"` character, indicating that the
the same phrase is used as in the previous parameter. So our
example above might also look like this:
\begtt \typosize[8.5/11]
\vdef {took the kingdom} % BBE
      {possessed the kingdom} % Jubilee2000
      {take possession of the kingdom} % NETfree
      {possessed the kingdom} % UKJV
      {"} % RNKJV    
      {"} % Webster
\endtt

\lookout
 Quotation marks in place of a phrase cannot be used in definitions of `\ww` (see~chapter~\ref[ww]) that precede `\Note`. They only work in the `\vdef` definition in the variants file, to be applied for the entire Bible. 


\secc[ww] Variant phrase declarations for pairing notes with text

The \`\Note` command may be immediately preceded by a declaration of the search
phrase by variant translations using \`\ww` (this is an abbreviation for watchword).
The \`\ww` command has as many parameters as there are translation variants declared by
\`\variants` command, and these parameters can be simple (in the format
`{<wanted-phrase>}`) or compound (in the format
`{<search-phrase>}={<what-to-print>}`). The immediately following `\Note`
will then ignore its parameter for the search phrase and use the parameter from \`\ww`
corresponding to the language variant being processed. For example:

\begtt
\ww
{if you do not make clear to me the dream and the sense of it} % BBE
{if ye will not make known unto me the dream with its interpretation} % Jubilee2000
{If you do not inform me of both the dream and its interpretation} % NETfree
{if all of you will not make known unto me the dream, with the interpretation thereof} % UKJV
{if ye will not make known unto me the dream, with the interpretation thereof} % RNKJV
{if ye will not make known to me the dream, with the interpretation of it} % Webster
\Note 2:5 {If you do not tell me what my dream was and interpret it.} % NIV; ignored, could be empty.
Nebuchadnezzar formulated a plan for testing his advisors. If they could not relate the dream back to him he would have no confidence in their interpretation (see <"v." 9>). 
\endtt

The `Note 2:5` given here searches the text in verse 2:5 for \"if you do not make clear to me the dream
and the sense of it" when a variant translation of the BBE is being processed, and looks up the text
\"if ye will not make known unto me the dream with its interpretation" when the Jubilee2000 translation variant is being processed.
The example assumes that six `\variants` have been declared using the `\variants` command
translation variants in the specified order. 

You can also specify a different phrase
for searching and for printing in a note, as shown in the following example:

\begtt
\ww {head}={head ... gold, ... breast and arms ... silver, ... middle and sides ... brass, legs ... iron, ... feet in part of iron and in part of potter's earth} % BBE
    {head}={head ... gold, ... breast and arms ... silver, ... belly and thighs... brass, legs ... iron, ... feet  part of iron and part of baked clay} % Jubilee2000
    {head}={head ... gold, ... breast and arms ... silver, ... belly and thighs... brass, legs ... iron, ... feet  part of iron and part of clay} % NETfree
    {head}={head ... gold, ... breast and arms ... silver, ... belly and thighs... brass, legs ... iron, ... feet  part of iron and part of clay} % UKJV
    {head}={head ... gold, ... breast and arms ... silver, ... belly and thighs... brass, legs ... iron, ... feet  part of iron and part of clay} % RNKJV
    {head}={head ... gold, ... breast and arms ... silver, ... belly and thighs... brass, legs ... iron, ... feet  part of iron and part of clay} % Webster
\Note 2:32-33 {} Moving from the head to the feet of the image, there is a decrease not only in the  weight of the materials but also in its value. The image was clearly too heavy with fragile feet. It is an illustration of the fate of all human kingdoms and civilizations:  at the end, each of them will collapse by its own weight.
\endtt

The search and replace phrases are used exactly as they are written in
parameters of the \`\ww` command. This does not apply to \`\Note` notes, which do not have
\`\ww` in front of themselves. Then when using

\begtt
\Note <string> {<search-phrase>} <text> <blank line>
\endtt
or
\begtt
\Note <string> {<search-phrase>}={<what-to-print>} <text> <blank line>
\endtt
the <search-phrase> is first transformed by the data from `\vdef`. Only if
this data does not exist for the search phrase, <search-phrase> is being used as it is.

\recommended 
For clarity, it is worthwhile to have each translation on a new line, and to label it after the commenting percentage, so that we know where what belongs without groping.
The last line of the `notes-*.tex` file for a particular book that \TeX\ loads should contain a single `\endinput` command.
Whatever follows below this instruction on subsequent lines will not be seen by \TeX. (But don't confuse it with `\end` or `\bye` so that \TeX\ doesn't stop running at this point but continues reading
other files.)

So under `\endinput` we can have a few lines ready, e.g. in this form:

\begtt
\ww {}={} % BBE
    {}={} % Jubilee2000
    {}={} % NETfree
    {}={} % UKJV
    {}={} % RNKJV
    {}={} % Webster
\Note 1:1 {}={}
\endtt

and then just copy these lines in place of the new note, edit the chapter number and verses after `\Note`, or delete `={}` where they are not needed. This way you don't lose track of where
which phrases belong where, whether you write them out manually or paste them from a Bible program or online resources.


%%%%%%%%%%%%%%%%%%%%%%%%%%%%%%%%%6.3

\secc[switch] Text process branching by translation variants

%\mnote{\code{\\switch}}
You can use the \`\switch` command to branch the text processing in
depending on the set value of the \`\tmark` parameter, i.e., depending on
the language variant currently being processed. The command has the following syntax:
\begtt
\switch {<list of variants>} {<what to do>}%
        {<list of variants>} {<what to do>}% ... etc.
        {<list of variants>} {<what to do>}
\endtt
The pairs `{<list of variants>} {<what to do>}` can be given as many times as you like.
After each pair `{{<list of variants>}{<what to do>}` (except for the very last pair) must be followed immediately and without spaces by another such pair,
Therefore, when moving to the next line, write a percent sign after the closing parenthesis to cover the gap from the end of the line. Gaps at the beginning of the next line do not matter.
You can understand the percent sign after the pair as “next pair continues”.

The `<list of variants>` is a single translation variant or a list of translation variants separated by
a comma and no spaces. The \TeX/ then works as follows:
If a variant defined by the `\mark` parameter occurs in the `<variant list>`, the
the following `<what to do>` is executed. If there is no such variant,
the following `<what to do>` is skipped. Example:
\begtt
\switch {BBE}         {water-door of the Ulai}% BBE
        {NETfree}     {Ulai Canal}%             NETfree
        {Webster}     {river Ulai}%             Webster
        {Jubilee2000,UKJV,RNKJV} {river of Ulai} % UKJV, RNKJV, Jubilee2000 
 \endtt
The example shows how to print the name of Ulai river, depending on
the language variant being processed.

Once \TeX/ finds a match and does `<what to do>`, then the possible following
entries within the same `\switch` command are skipped. Furthermore, the rule is that
if `<list of variants>` is empty, then `<what to do>` is always executed, if
not skipped according to the previous rule. So an empty `<list of variants>` 
at the end of the `\switch` parameter pairs is evaluated as “all other
cases”. The example above can also be written as follows:
\begtt
\switch {BBE}         {water-door of the Ulai}% BBE
        {NETfree}     {Ulai Canal}%             NETfree
        {Webster}     {river Ulai}%             Webster
        {}            {river of Ulai}           % UKJV, RNKJV, Jubilee2000 
\endtt

The `\switch` command can be used not only for single phrases within notes
`\Note`, but also to entire sections of input text containing, for example, `\Note`,
several `\Note` notes, several definitions, etc.

The `\switch` command cannot be used in the parameters of other macros. Only the
command `\x/<phrase>/` works there.


%%%%%%%%%%%%%%%%%%%%%5


\secc[renum] Renumbering verses according to translation variants

%\mnote{\code{\\renum}}
Some translation variants have different verse numbering. In this case, the
the \`\renum` command can be used as follows:

\begtt
\renum <book-abbreviation> <default-chapter-number>:<default-verse-number> = <translation> <chapter-number>:<verse-number-from>-<verse-number-to>
\endtt
where `<translation>` is the mark of a particular translation.

The `<default-chapter-number>:<default-verse-number>`  is replaced with 
<chapter-number>:<verse-number-from>-<verse-number-to> in `\def\tmark{<translation>}`
Such renumbering does not only apply to
this verse, but the entire range of verses defined by `<verse-number-from>-<verse-number-to>`.

For example, let's suppose that
in the book of Daniel, in verses 10:20 and 10:21 we want to comment on several phrases, such as “angel of Persia” and “true writings”. For some reason (not entirely clear) BBE has these phrases in oposing verses then the rest of the translations.

We number the notes according to the translation given as the first parameter of the definition of `\variants` in the `vars.tex` file, (in our example `BBE`) including references like “See note on 10:20.”
Renumbered translations change the note number according to the actual verse number that the note comments on, including the reference, which is then printed as “See note on 10:21.”

If the renumbering is to refer to a single verse, the identical 
`<verse-number-from>` and `<verse-number-to>`  should be given, as in the example below:

\begtt
\renum Dan 10:20 = Jubilee2000 10:21-21
\renum Dan 10:20 = NETfree 10:21-21
\renum Dan 10:20 = UKJV 10:21-21
\renum Dan 10:20 = RNKJV 10:21-21
\renum Dan 10:20 = Webster 10:21-21
\endtt



%After such a declaration, the final remark 
%\begtt \catcode`\<=12
%\Note 5:31 {Darius}={\x/Darius/Median} Some schools claim that this and other 
%   (<6:1>, <6:6>, <6:9>, <6:25>, <6:28>; <9:1>; <11:1>)     
%   The references to \x/Darius/ Medes in the book of \x/Daniel/ are historical errors.
%\endtt
%prints as follows:
%
%{\medskip\leftskip=\parindent\noindent
%{\bf 6:1 Darius the Mede} \ Some schools claim that this and other 
%(6:2, 6:7, 6:10, 6:26, 6:29; 9:1; 11:1)     
%references to Darius the Mede in the book of Daniel are historical errors.
%\medskip}

%Even such tidbits as shifting a number by a mere part of a verse can be dealt with.
%BDB Da 2:28 \"there is a God in heaven who reveals secret things" is in the SNC \"God who is in heaven reveals secrets." 2:27;
%So before the note on this phrase, we will put `Renum Da 2:28 = SNC 2:27-27`. 
%But then we also need a note on the phrase \"in the latter days" of the same verse, but -- world wonder -- the phrase
%\"in the latter days" is already
%is in the SNC in verse 2:28! We don't need to speculate why the SNC translators try such tricks on us by sending half a verse under a different number than everyone else;
%it is enough that we know how to handle it: 
%We write ``renum Da 2:28 = SNC 2:28-28'' before this new note, and everything works as it should: Where the numbering diverges, it renumbers; where it
%it matches, it stays the same.
%%%%%%%%%%%%%%%%%%%%%%%%%%%%%555

\sec[links] Methods of creating hyperlinks

A reference is a section of text by which the reader, even after printing it, can tell to what other place in the text (or in the~internet) one can look at. Thus, it typically contains a numerical indication of page or the number of a chapter, section, etc. In addition, if the reader is working with a PDF viewer, then this section of text is {\em active}, i.e., when the mouse hovers over
this text you can click on it and the PDF viewer will go to the specified place in the document (or to the Internet).

The Bible is invariably
structured text. It contains (in the Protestant canon) 66 books with established markers for those books,
each book has its chapters numbered from one and each chapter has verses
numbered from one. So there is no need to have \TeX/ generate these numbers
automatically (as it does when typesetting, say, a technical text that is broken down into
chapters and sections), and thus there is no need to use reams in the source
in the source document (which \TeX/ assigns to the generated numbers during processing) and to reference the
these funnels, as described in section 1.4.3 of the \OpTeX/ documentation. It is
much more efficient to refer directly to a specific place in the Bible that already has
for many centuries now, with a fixed book, chapter number and verse number.
(The Archbishop of Canterbury, Stephen Langton, in the early 13th century, when he was teaching at the University of Paris [and not yet an archbishop]
divided the Bible into chapters. 
Then, in the mid-16th century, the French printer Robert Estienne divided the New Testament into verses and added the Old Testament,
which Jewish scribes had centuries earlier divided into verses, but cunningly not into chapters.
Since 1553, when Estienne published the first French Bible so numbered, we have used this system to this day.)


We write references to a specific place in the Bible between \code{<} and \code{>}.
these characters is printed as written (with the exceptions noted below). 
But \TeX/ must be able to interpret the reference correctly in order to
make it active with a possible click-through to the correct place in the Bible. This is done by
for this purpose, which is the subject of this section.

Basic rule with full details

The reference between \code{<} and \code{>} is of the form `"<text>" <data>` or just
`<data>`. However, the complete `<data>` is of the form
`<book> <chapter>:<verse>`. Here `<book>` is the abbreviation of the book (it must be followed by
space), `<chapter>` is the chapter number, and `<verse>` is the verse number.
Example:

\begtt \catcode`<=12
... see also verse <Jr 8:13>
... see also <"verse" Jr 8:13>
\endtt
In the first case it prints ... see also verse {\Blue Jr 8:13} and in the second case
... see also {\Blue verse Jr 8:13}. The active (allowing click-through) will be in both cases
the area marked in blue here and the click-through leads to Jr 8:13.

\secc Link specifier

The link ending character `>` may be followed closely by a link specifier, which
is one of the letters:
\begitems
* `n` ... refers to a note,
* `g` ... refers to a gloss (which will come into play if another version of \OpBible/ with double-column typesetting of the base text is ever created),
* `a` ... refers to an article,
* `i` ... refers to the introduction.
\enditems
The link specifier is not printed, it is just internal information where it should
the active link (i.e. ~mouse click). If it is not followed by a terminating
character `>` is followed by any of these specifiers, it is a link to the verse
(this is probably the most common case).
Example of a note reference:

\begtt \catcode`\<=12
... see <"note on" Jr 7:4>n for more information.
\endtt
prints ... see {\Blue note to Jr 7:4} for more information.
The click-through leads to the first note to ~Jr~7:4, not to the verse itself.

In the case of a reference to an article (specifier `a`), the full entry has the format
`<book> <chapter>`, i.e., the verse information is missing because the articles can be understood as introductions
to chapters. In the case of a reference to the introduction to a book (specifier `i`), the full
entry is in the format `<book>` (lacking both chapter and verse information) because it
are book introductions. In~other cases, the full entry has the format as it was
discussed in section~\ref[uudaj] except as described in~section~\ref[vudaj]

\secc[vudaj] Exception for the format of the full entry for some books

Books Abd, Fm, 2Jn, 3Jn, and Ju are not divided into chapters. In a reference to a verse, note
to a verse or a gloss to one of these books, the chapter information is missing
and the format of a complete reference looks like this: `<book> <verse>`. To let \TeX/ know that
to apply this exception, the list of abbreviations for these books needs to be defined in
macro \`\nochapbooks`. For example, the `Cze-books.tex` file says
\begtt
\def\nochapbooks{Abd Fm 2Jn 3Jn Ju}
\endtt
Since there are different book abbreviations for different languages, this macro is needed
to define this macro depending on the language used.


\secc[bored] Incomplete

Sometimes the reader can determine the location of a verse from the context, so `<data>` may not be complete. In ~{{incomplete data}}, there may be a missing `<book>` or
`<book> <chapter>:` or all of it. For example:

\begtt \catcode`<=12
... we also see an analogy in <"verses" Jr 8:13>, <9:7> and <11:3>
... see verses <Jr 8:13>, <15>, <17>
... see all <"verses" Jr 8:13>--<22>,
... (cf. <Jr 8:13> and <"his note">n).
\endtt
Printed:
... We also see an analogy in {\Blue verses Jr 8:13}, {\Blue 9:7} and {\Blue 11:3}
... see verses {\Blue Jr 8:13}, {\Blue 15}, {\Blue 17}
... see all of {\Blue verse Jr 8:13}--{\Blue 22}.
... (cf. {\Blue Jr 8:13} and {\Blue note}).

The entries `9:7`, `11:3`, `15`, `17`, `22` and the last blank
in these examples are incomplete. The reader knows that they refer to
the book of Jeremiah and that, where no chapter is given, they refer to chapter 8 of
Jeremiah. In the last example with a blank, the reader knows that it is a note
k~Jr~8:13.
links (to be clicked on) because the incomplete entry takes
unspecified information from the previous entry. This rule applies
locally to a single text object: note, article, introduction, etc. If the very first entry in
the very first entry in a text object is incomplete, the unspecified information is replaced with
by the abbreviation of the book currently being processed, or by the number of the current chapter,
or verse.

If an incomplete entry is prefixed with `\`, the unspecified information is taken from
the book or chapter or verse currently being processed, regardless of which entry
precedes. For example:
\begtt \catcode`<=12
\Note 4.5 {} The idea is repeated in <Jr 4:5>. But you will also find it mentioned in \<8:3>.
\endtt
Here the reference {\Blue 8:3} leads to verse 3 of chapter 8 of the actual book. If there
the `` sign was not there, then this reference would lead to 8:3 of Jeremiah.

An incomplete entry is printed as is, as incomplete. The above rules on its
are only done internally so that the active link will work properly after
mouse click.

There is another different format for incomplete data: the verse number is missing. This
is indicated by the presence of the book abbreviation and the absence of a colon and
it is not a book from the `\nochapbooks` exception list. Thus, the entry is in the form
`<book> <chapter>` (separated by a space). In this case, \TeX/ will create
internal reference to the first verse of the chapter. Examples:
\begtt \catcode`<=12
... see <"chapter" Da 7>
... Joseph's story (<Gn 39-41>)
\endtt
In the second example, the range of chapters is given and \TeX/ creates an internal link to
the first verse of chapter 39. Compare also with section~\ref[verse range].


\secc[verse range] Format for the verse range and for the section in the verse

In each entry, it is possible to have `<verse>` instead of `<chapter>:<verse>`
to write a range of verses in the format
`<from>-<to>`. The \TeX/ will create an internal link to the first verse of the range only and
turns the hyphen (character `-`, ASCII 45) in the range into a dash. Examples:

\begtt \catcode`<=12
<Jr 8:3-7>,
<Jr 8:3-9:5>,
<3-7>,
<8:3-7>.
\endtt
For example, the first link in this example prints as {\Blue Jr 8:3--7}
and only offers a link to Jr~8:3.

Sometimes you need to link to a section of a verse, not the whole verse. This is done
by appending a letter immediately after the verse number. For example,
\begtt \catcode`<=12
... see <Da 9:11b>
\endtt
You can append such a letter to both complete and incomplete references. For the purpose of
hyperlink, these letters are ignored but printed. Thus, the example given
prints ... see {\Blue Da~9:11b}, but the link goes to Da~9:11.

\secc Concealment of data

When none of the rules are sufficient to create an internal link from a link
listed above, you can make everything in the link
to print, enclose it in `"..."` and the subsequent entry to create the internal
link, to hide. To do this, just append the character `"` just after the closing quotation mark
underscore `_` followed by the data. This entry is not printed. For example,
\begtt \catcode`<=12
<"First Book of Samuel"_1Sa 1:1>
\endtt
prints only the text {\Blue First Book of Samuel}, which internally refers to the first
verse of this book.

\secc[Renum] Renumbering the reference

If a link points to a verse that has a different translation in a particular language
numbering specified by the `\renum` macro, then enter the reference according to
the default numbering and \TeX/ will recalculate it itself according to the data specified in
`\renum`. It prints the recalculated data and uses it for the internal reference.
Cf.~\ref[renum]. 

\secc Data reduction

You may want to write the complete data in parentheses for the reference and require it to be automatically 
reduction to incomplete if they refer to the current book.
If they do not refer to the current book, the entry will remain complete. You can do this by
by adding a \`\re` before the opening parenthesis of the entry (the eventual reduction rule 
applies only to this single entry) or by using the \`\reduceref` command. If you
you use it in a note (or in a \TeX group), the reduction rule is applied 
for all subsequent complete entries of that note (of that \TeX/group). 
When used in a rate declaration,
the eventual reduction rule is applied to all complete data in the document. 
Example:

\begtt \catcode`<=12
\re<"verse" Da 7:3>
\endtt
is printed as {\Blue verse Da~7:3} if this reference is given outside the book
Daniel. However, when this reference is given in the book of Daniel, it is printed
only {\Blue verse~7:3}, which internally refers to Da~7:3. 

The link reduction rule set by \`\reduceref` can be turned off 
with the \`\noreduceref` command. From there to the end of the note (\TeX/ groups)
links behave as if `\reduceref` were not turned on.

\secc[bookajinak] The marker for a book can be printed differently

If the bookmark is declared differently in different translations using
`\vdef`, then use only the <book> entry in the ~references according to the first
variant. However, if any alternative translation is set using `\tmark`
variant, the link will eventually print according to the `\vdef` entry of that translation variant.
Internally, however, links are linked according to the base variant. This property
This feature is useful if you are declaring `\vdef` for tags `1Pa` and `2Pa` (Paralipomenon)
alternative text for `1Let` and `2Let` (Chronicles) in variant B21 (Bible for the 21st century). When then
you type the reference \code{<"see" 1Pa 2:3>}, you get {\Blue see 1Pa~2:3} in the usual
translation variants, but it will print {\Blue see 1Let~2:3} if you use
translation variant~B21.

\secc[fudaj] Failed links, i.e. links to a non-existent place

If a reference is made to a non-existent verse or a non-existent note, then
there are two possibilities. If it is a reference to a book that is deliberately not printed
(because, for example, we are working with a test copy of only selected books of the Bible,
see also \`\printedbooks` in section~\ref[main]), then the link is indeed colored,
as if it were active, but it is not, and the \TeX/em processing log
no warning appears. However, if the link points to a non-existent verse or
note of the printed Bible, then the link is active, the click goes to
the last page of the PDF file, and a warning appears in the log that the link
is incorrect. However, the first time \TeX/ is processed, all
links and there is a large number of warnings about incorrect links in the log.
Only when \TeX/em is next processed are links pointing to
existing site are correctly interlinked.


\secc Link Tracing

By default, detailed link tracing is enabled in the log file.
Turn this off with \`\notracinglinks` and back on with \`\tracinglinks`.
In addition, use \`\tracingouterlinks` to disable the suppression of link warnings on
non-existent books, allowing the log to find any non-existent links from
due to a typo in the bookmark.

\secc Chapter and book links

To link to a book with a link to the beginning of that book, write
\`\cref``[<book>]`, for example, `\cref[Gn]` will print {\Blue Gn} with a link to
the beginning of the book of Genesis. Similarly, `\cref[Gn 2]` will print {\Blue Gn~2} with a click through
to the beginning of the Day of Rest chapter. If you want to print something else, then for
`[...]` must be closely followed by `{<text>}`, where `<text>` is the text that
to be printed and become the active link. So, for example
`\cref[Gn 2]{Rest Day}` will print {\Blue Rest Day} with a link to
the corresponding chapter.

\secc Links to pages

You can place an unprintable page link target in your text using \`\pglabel``[<glue>]` and
then you can link to the page using \`\pgref``[<glue>]`. In place of
`\pgref[<glyph>]` will print the number of the clickable page.
Similar to `\cref`, you can use `\pgref[<picker>]{<text>}`
to print a `<text>` other than the page number that clicks through to the place where
where the `\pglabel[<glue>]` is located.



\sec[maps] Maps, images and their legends

\secc[map-variants] Translation variants

Just as you can change the wording of a search phrase in a note to match the actual base text, you can prepare 
the description of the map (and similar graphical objects showing expressions that vary in different translations) to always match the current version of the Bible.

Assuming that we have a blank map that we want to describe with its legend, any text is placed over an image as follows:
\begtt
\puttext 5mm 62mm{{{\bi Mediterranean}}}
\endtt

The first dimension (`5mm`) determines the horizontal offset on the x-axis, the second the vertical offset on the y-axis, with the coordinate `0mm 0mm` being the bottom left corner of the image.





\medskip
\line{\picw=150pt \inspic{images/BKR-Da-intro-map.png} \hss\inspic{images/CEP-Da-intro-map.png} \hss\inspic{images/B21-Da-intro-map.png} } 
\medskip
\leftline{Map from the Introduction to Daniel:  BKR \hss CEP \hss B21}
\medskip
For example, you can notice how in the sample above not only the phrases “midnight king” and “noon king”(BKR) change to “north king” and “south king” (CEP) and “south king”
and “King of the North” (B21), which is a common active link as in the notes, but also the map description itself , e.g. 
The name of the city of Susan (just above the Persian Gulf) from the King's name to Shushan in CEP and to Susa in B21.
metamorphoses to Sinear and to Babylonia. The definitions of `\vdef` in the `Cze-vars.tex` file take care of this. 

The image must be loaded before the description can start, and everything in it must be inside the definition of `\insertBot`. 
The above sample has been written as

\begtt
\insertBot {Daniel's Remote Viewing Realm}[map](){
 \inspic{fertile-crescent-crop.pdf}% blind map 
  \Heros \cond \setfontsize{at 9pt}\rm %font
  \vskip-1mm 
\putstext 2mm 108mm {\top{\hsize6.5cm %box width\top
                     \baselineskip10pt %line spacing inside the box
                     \noindent %space saving, no need to indent
\leftskip=3pt \rightskip=3pt %how much the semi-transparent shadow will overlap the text 
  Soon after the death of Alexander the Great ... until the defeat by Rome in 63 B.C.}%end\vtop
  }%end \putstext
  \LMfonts\sans \setfontsize{at9pt}\rm
  \puttext 145mm 29mm {<"Acts 2:9"_Acts 2:9>}
  \puttext 145mm 32.5mm {<"Ez 32:16"_Ez 32:16>}
.
.
.
  \puttext 2mm 5mm{{{{Heros \setfontsize{at 7pt}\it Satellite Bible Atlas,\rm W.Schlegel}}
  \puttext 2mm 2mm{\Heros \setfontsize{at 7pt}\rm Used with permission.}
  }%end \insertBot
\endtt

{\bf What to watch out for:} There must not be a blank line inside `\insertBot`.

Command \`\putBot` `<verse number> {<title>} [<slash>] (<commands>) {<code>}`
behaves similarly to \`\insertBot`, but inserts the image or diagram not on the current
page, but to the page where the verse specified by its <verse-string> is located.
Such commands can be used in an annotation file, for example at the end of the file.


\secc[town] Macro `\town` for the town symbol on the map


The towns of Jerusalem, Babylon, Tolul Dura, Susan and Ur are visible on the map as tiny circles with a red center and black perimeter.
The properties of this ring can be set with the macro \`\townparams`, whose default values are as follows:

\begtt
\def\townparams{
   \hhkern=.8pt % radius of the sphere
   \lwidth=.5pt % contour line thickness
   \fcolor=\Red % ring color
   \lcolor=\Black % contour line colour
}
\endtt

The macro \`\town` itself places this marker with coordinates, similar to the macro `\puttext`, but without additional text, e.g.

\begtt
  \town 101.5mm 53mm %town Babylon
\endtt

\secc [slope] Tilted text

In the example we have the name of the city of Jerusalem printed at an oblique angle so that it does not clash with “king of the South” and can be seen clearly.
This can be achieved by adding additional parameters to the `\puttext` definition:

\begtt
\puttext 48mm 55mm {\c[-40/\kern1pt\pdfrotate{0}]{Jerusalem}}
\endtt

The number `-40` was used to tilt the inscription. For further details see~\ref[maptitles].



\secc[maptitles] Inscriptions along the curve

Some of the inscriptions on the map require “stretch” according to the terrain, especially the names of large areas, in this case the Ptolemaic and Seleucids, or the Persian Gulf, or perhaps the more minor
names of rivers (Euphrates, Tigris, Nile). 
We will change the additional parameters to `\puttext` as we have already seen when placing Jerusalem on the map in the v~\ref[slope].

The `\kern` command determines the spacing between letters; the digits in the `\pdfrotate` definition determine the strength of the curvature. A negative value bends the lettering concavely (like a rainbow), a positive value convexly (like a bowl).
The dynasties of Alexander's descendants are written on our map as follows:

\begtt
\puttext 62mm 70mm {\c[10/\kern7pt\pdfrotate{-1}]{SELEUKOVCI}}
\puttext 2mm 37mm {\c[0/\kern4pt\pdfrotate{2.5}]{PTOLEMAI}}
\endtt

If we needed a sign that waved in the shape of the letter S (it would be concave and convex at the same time), we would have to assemble it 
from two or more `\puttext` statements, glued together to look like one continuous text. 




\secc[bkgrnd] Partially transparent background of continuous text


In the \ref[map-variants] sample, the double continuous text in the map was originally placed in lighter rectangles prepared by the graphical editor (left image). The text then had to be inserted into these
hit the text; if it didn't work, one had to redo the image and try to hit it again.

\medskip
\centerline{\picw=150pt\inspic images/fertile-crescent-crop-old.pdf
\hss \inspic{images/Da-map-shadowed-text.png}%
}
\smallskip

It worked, but it was tediously laborious and time-consuming.
\OpBible/ offers a more imaginative solution. 
no pre-made areas. Before using `\putstext` for the first time, the level of transparency of the white shadow can be set by adjusting the default value of `\def\shadowparameter{.1}`
(right image). `\def\shadowparameter{1}` means a solid opaque white background; a smaller number means more transparency.
However, this value is then stored in the page-resources of the output PDF and is used the same on all subsequent pages, so it cannot be changed and have it in different places in the same document
different on different pages. If perhaps there should be an unexpectedly excessive demand for the ability to change the transparency level on the fly, this may be an incentive for implementation in a possible future version of \OpBible/.
For the moment, we did not find it necessary to complicate the macros by creating more and more page-resources, so the user should be satisfied with the option to set the transparency of the shadow under the text
on maps uniformly for the entire Bible.  



%\medskip
%\centerline{\picw=150pt \hss\inspic{images/Da-map-shaded-text.png} } 
%\medskip

\sec Timeline inclusion tools

\secc[spanimage] Image or text over two pages

If we want to insert an image or text across two pages in an open double page, we can
use
\begitems
* \`\insertSpanImage`: inserts a prepared PDF image, can be used in the introduction of a book,
* \`\insertSpanText`: insert text (for example a timeline), can be used in the book introduction,
* \`\putSpanImage`: insert a prepared PDF image, anchored relative to the number
   chapter and verse, can be used in the notes file,
* \`\putSpanText`: like `\putSpanImage`, but inserts text instead of an image.
\enditems
The `\insertSpanImage` and `\insertSpanText` commands place the image or text 
at the bottom of two pages according to the following rule. Suppose that
the command itself is executed when \TeX/ creates the current page with
number $c$. Then
\begitems
* if $c$ is even and the image or text fits the current page in portrait,
  it will be placed on pages $c$ and $c+1$,
* if $c$ is even and the image or text does not fit on the current page,
  it will be placed on pages $c+2$ and $c+3$,
* if $c$ is odd, the image or text will be placed on pages $c+1$, $c+2$.
\enditems
This ensures that the image or text will always be visible on the double page
of an open book.

The `\putSpanImage` or `\putSpanText` commands work according to the same rule,
only the number $c$ in this case corresponds to the page number on which the
the beginning of the verse specified in the command parameter.

These commands have the following parameters:

\`\insertSpanImage` `{Title} [<column>] (<parameters>) {<filename>}`
Here, <Title> is used in the header of the image and <snapshot> can be set,
to refer to the image (and thus create an active link) using `\ref`.
If you don't want to use the funnel, leave the parameter empty: `[]`.
Furthermore, <parameters> can specify how the image is placed, typically this
parameter is empty: `()`. Finally, <filename> is the name of the image file
including the extension. Typically this is a PDF file, i.e. it has the extension `.pdf`.

\`\insertSpanText` `{Title} [<column>] (<parameters>) {<text>}`
The parameters are the same as `\insertSpanImage`, only the last parameter is different.
It contains the text to be printed across two pages. Typically there might be
a set of `\timeline`, `\timelinewidth`, `\arrowtext`, `\tlput` commands can be used,
`\tline`, `\tlines` to create a timeline, see section~\ref[timeline].

\`\putSpanImage` `<chapter>:<string> {Title} [<column>] (<parameters>) {<filename>}`
In addition, there is a <chapter>:<verse> specifying the verse whose beginning is on page $c$
and the image is positioned according to the rule above. The other parameters are the same as
for `\insertSpanImage`.

\`\putSpanText` `<chapter>:<verse> {Title} [<image>] (<parameters>) {<text>}`
behaves the same as `\insertSpanText`, but with the addition of the <chapter>:<string> parameter
has the same meaning as in the `\putSpanImage` command.


\secc[timeline] Commands to create a timeline

These commands can be used in the <text> parameter of the \`\insertSpanText` or \`\putSpanText` commands.

First, you need to specify the number of years (or other units) for the full width of the timeline. 
All other data will be entered in these units. In the following text
we will refer to these units as years.
Set the timeline parameters using \`\timeline` and \`\timelinewidth`:
\begtt \catcode`\<=13
   \timeline <number of years>
   \timelinewidth <width of the axis>
\endtt
For example, after
\begtt
   \timeline 500
   \timelinewidth 25cm
\endtt
the axis will be 25cm wide and 500 years will be included in that width, so one year
will represent a width of 0.5 millimeters. However, it is more usual to specify the width of the axis
as some fraction of the total width of the page (or double-sided image, if using
axis in \`\insertSpanText` or \`\putSpanText`). For example:
`\timelinewidth=.95\hsize`.

The timeline is built line by line. Commands for text or line segments that
are on the same line are written below each other in any order. To move to the next
line is the \`\vskip` `<dimension>` command, while the dimension can be specified in
in multiples of the line height using the unit \`\l`. i.e. `\vskip 1.5\l`
means a shift of one and a half lines down.

To insert text, use \`\tlput` `<flag> <position> <space> (<setting>) {<text>}`.
The <flag> parameter can be `a` if we want the text to be above the current
rate position, or `b` if we want the text to be below the current rate position.
When the <flag> is `a`, multiline text extends upwards and when `b` extends downwards.
The <position> parameter is the location on the timeline (in years) to which the text should be attached. 
From this point, the text will flow to the right if <space> is specified
by the \`\rlap` instruction, it will flow to the left if the <string> is specified by the
\`\llap`. Finally, if <query> is empty, the text will have centered lines and
<position> then corresponds to this line center. The `<setting>` parameter can be
be empty, i.e. `()`, or it may contain font settings, font color, etc. for
following <text>. A <text> contains the text to be printed. In the case of
multi-line text, separate the lines with \`\cr`. Example:
\begtt
   \tlput b 25 (\it) {Abraham\cr is 100 years old}
\endtt 
The text has two lines, their common center is below point 25 of the timeline.

\`\tline` `<from>..<to>` creates a horizontal line starting at <from> and
ending at <to>. The data are in years.

\`\tlines` `<w1>|<w2>|<w3>|` (and possibly others) inserts short vertical lines on the axis.
The number of `<w>|` parameters can be arbitrary, each representing one
vertical line, and the numbers between them indicate the distance between adjacent lines in
years. For example
\begtt
\tline 0..100
\tlines 20|20|20|20|20|
\endtt
creates a 100th horizontal line and vertical lines on top of it to denote 20,
40, 60 and 80 years.

\`\arrowtext` `<from>..<to> (<setting>) {<text>}`
prints a horizontal line from <from> to <to> (data in years)
and the middle of the line is broken and <text> is written at that point. At
the edges of the line are arrows pointing outwards from the line.


\sec Page formatting variants

By default, a single-column typeface is set for the main text of the Bible
and for book introductions and annotations. Two-column type is set for
verse notes.

By default, chapter numbers are capitalized and in the outer margin. In the outer margin are also
enlarged quotation marks attached to quotations. The command \`\normalchapnumbers`
command changes this setting: chapter numbers are then inserted in the left
upper corner of the first paragraph and the enlarged quotation marks are removed.
The outer margin is then shrunk because there is no more print material in it.  

By default, each \`\Note` occupies a new paragraph in a two-column
typeface. By using `\mergednotes` it is possible to ensure that all notes referenced
to a single verse are always combined into a common paragraph. But it comes at a cost
of not ensuring that the occurrence of phrases in the main text is on the same page
as the phrases in the note. The entire paragraph of the linked note will only link to
the beginning of the verse, i.e. the beginning of the verse and the beginning of the note to the verse are on the same
page, the phrases themselves are not searched.

Other page formatting options are still in the planning stages and are not in this version of \OpBible/
implemented.


\sec Error search options

It may happen that you make a typo in the `notes-*`, `intro-*` files,
`articles-*`. If you had included the file directly in the whole Bible processing,
TeX would report an error at a completely different point in time than when it read the file and
tracing back such an error is very difficult, and typically the error
occurs at a different stage of processing and thus is often reported at a very
incomprehensible and misleading.

However, it is possible to process newly written files directly first, but without
but not linked to the main text, and in this mode errors are reported more directly.
To search for errors directly, use \`\checksyntax` <file list> `{}`
Here the file list are the names of the files checked without the `.tex` extension,
so you can have, for example, at the end of the main file:

\begtt
\checksyntax intro-Dan articles-Dan notes-Dan {}
\endtt

performs a direct check of the enumerated files. The output is the text of these
files without any formatting emphasis. Syntax errors in the files are
but will show up in a straightforward manner.

Note: using \`\checksyntax` disables the \`\processbooks` command, which in
does nothing, so the base text of the Bible is not loaded at all.


\secc Generating default files `notes`, `fmt`, `intro`

If the files `notes`, `fmt`, `intro` do not exist, \OpBible/ does not get angry and
behaves as if they were empty. But you might want to have these
files for all the books of the Bible already prepared in the appropriate directories and
with a short introduction. Templates for such files can be
write them in a separate file (e.g. `templates.tex`) and then
file using the `optex templates` command. This will generate the default files
according to the specified templates for all the books of the Bible.
If a file with the specified name already exists, it is retained (i.e., it is not
overwritten by the default file) and a warning will appear on the term and
in the `log` file.

First you need a `books.tex` file with the names and abbreviations of all the books of the Bible,
as mentioned in the~\ref[books] section. This can be generated, for example
from Sword using the `mod2tex` script, see section~\ref[txs].

In the `templates.tex` file, the macro must first be loaded using `\input`.
`opbible.opm`, then the `books.tex` file, and then you can specify
file templates with the `\filegen` command as follows:

\begtt
\filegen {<file-name>}
<file-content>
\endfile
\endtt
%
while <filename> must contain and <file-content> can contain a double character
`@@`, which is automatically replaced by the book abbreviation <bmark> specified in `bible.tex` and generated
as many default files as there are book titles declared in `books.tex`.
For example, `templates.tex` might look like this:

\begtt
\input opbible.opm
\input books.tex

\filegen {intro-@@.tex}
% Book Introduction @@
\endfile

\filegen {notes-@@.tex}
% Book notes @@

%\ww {} {} {} {} {} {} {}
% BKR PSP CSP CEP B21 SNC
%\Note 1:1 {} text terminated with a blank line
\endfile
\endtt
%
The `optex templates` command will generate 66 files in this case
`intro-1Jn.tex`, `intro-1kor.tex`, etc. with the specified single line
and 66 `notes` files with specified five-line content.

You can also use the `@@@` triple character in the <file content>, which is interchanged from
the full title of the book. So after specifying `% Book Introduction @@@` you will get
files with the text `% Introduction` {\to the book of Genesis}, then
`% Introduction to the Book of Exodus` etc.

In the example, the file names are chosen so that they are generated in the current
directory. If you have prepared subdirectories in this directory corresponding to
names, it is possible to generate files directly into them, for example, write
`\filegen {intros/intro-@@.tex}`.

To generate a different list of files than the <bmark> list
listed in the `books.tex` file, define `books.tex` after reading
macro \`\genbooks` and your chosen shortcuts. For example, after
\begtt
\def\genbooks {Gn Ex Lv Nu Dt Joz}
\endtt
%
the `\filegen` command will only generate templates for the five books with the specified
abbreviations. You can also define `\genbooks` before each `\filegen`, then
you'll have different book abbreviations for different subfile types.


%This section is probably useless so I'm deleting it. I suppose we'll ship something like a "blank project" as part of the distribution, with files that don't contain theological material (except as samples), but examples of usable definitions, perhaps with comments on what they do.
%
%\sec[examples] Examples
%
%\secc[sample-main] Example of the main file `main.tex`
%
%
%\begtt
%\load[opbible] % macros OpBible
%\cslang
%\def\shadowparameter{0.075}
%%\checksyntax notes-Da intro-Da articles-Da fmt-BKR-Da {} 
%%\normalchapnumbers  
%%\mergednotes        
%%\let\notecolor=\relax % disables \Red search phrases %
%
%%\input hebrew % Hebrew phrases are declared here
%\input greek % Greek - "-
%
%% Variants of translation:
%\def\tmark {BKR} % Bible royal
%%\def\tmark {PSP} % Pavlik's Study Translation
%%\def\tmark {CSP} % Czech study translation
%%\def\tmark {CEP} % Czech Ecumenical Translation
%%%\def\tmark {B21} % Bible for the 21st century
%%\def\tmark {SNC} % Word for the Journey
%
%\input {Cze-vars.tex} % Variants of translation
%\BookTitle Gen  Gn {První Mojžíšova (Genesis)}
\BookTitle Exod Ex {Druhá Mojžíšova (Exodus)}
\BookTitle Lev  Lv {Třetí Mojžíšova (Leviticus)}
\BookTitle Num  Nu {Čtvrtá Možíšova (Numeri)}
\BookTitle Deut Dt {Pátá Mojžíšova (Deuteronomium)}
\BookTitle Josh Joz {Jozue}
\BookTitle Judg Sd {Soudců}
\BookTitle Ruth Rt {Rút}
\BookTitle 1Sam 1S {První Samuelova}
\BookTitle 2Sam 2S {Druhá Samuelova}
\BookTitle 1Kgs 1Kr {První Královská}
\BookTitle 2Kgs 2Kr {Druhá Královská}
\BookTitle 1Chr 1Pa {První Paralipomenon (1. Letopisů)}
\BookTitle 2Chr 2Pa {Druhá Paralipomenon (2. Letopisů)}
\BookTitle Ezra Ezd {Ezdráš}
\BookTitle Neh  Neh {Nehemjáš}
\BookTitle Esth Est {Ester}
\BookTitle Job  Jb {Jób}
\BookTitle Ps   Ž {Žalmy}
\BookTitle Prov Př {Přísloví}
\BookTitle Eccl Kaz {Kazatel}
\BookTitle Song Pís {Píseň písní}
\BookTitle Isa  Iz {Izajáš}
\BookTitle Jer  Jr {Jeremjáš}
\BookTitle Lam  Pl {Pláč}
\BookTitle Ezek Ez {Ezechiel}
\BookTitle Dan  Da {Daniel}
\BookTitle Hos  Oz {Ozeáš}
\BookTitle Joel Jl {Jóel}
\BookTitle Amos Am {Ámos}
\BookTitle Obad Abd {Abdijáš}
\BookTitle Jonah Jon {Jonáš}
\BookTitle Mic  Mi {Micheáš}
\BookTitle Nah  Na {Nahum}
\BookTitle Hab  Abk {Abakuk}
\BookTitle Zeph Sf {Sofonjáš}
\BookTitle Hag  Ag {Ageus}
\BookTitle Zech Za {Zacharjáš}
\BookTitle Mal  Mal {Malachiáš}
\BookTitle Matt Mt {Matouš}
\BookTitle Mark Mk {Marek}
\BookTitle Luke Lk {Lukáš}
\BookTitle John Jn {Jan}
\BookTitle Acts Sk {Skutky apoštolské}
\BookTitle Rom  Ř {Římanům}
\BookTitle 1Cor 1Kor {První list Korintským}
\BookTitle 2Cor 2Kor {Druhý list Korintským}
\BookTitle Gal  Ga {Galatským}
\BookTitle Eph  Ef {Efezským}
\BookTitle Phil Fp {Filipským}
\BookTitle Col  Kol {Koloským}
\BookTitle 1Thess 1Te {První list Tesalonickým}
\BookTitle 2Thess 2Te {Druhý list Tesalonickým}
\BookTitle 1Tim 1Tm {První list Timoteovi}
\BookTitle 2Tim 2Tm {Druhý list Timoteovi}
\BookTitle Titus Tt {Titovi}
\BookTitle Phlm  Fm {Filemonovi}
\BookTitle Heb   Žd {Židům}
\BookTitle Jas   Jk {List Jakubův}
\BookTitle 1Pet  1Pt {První list Petrův}
\BookTitle 2Pet  2Pt {Druhý list Petrův}
\BookTitle 1John 1Jn {První list Janův}
\BookTitle 2John 2Jn {Druhý list Janův}
\BookTitle 3John 3Jn {Třetí list Janův}
\BookTitle Jude  Ju {List Judův}
\BookTitle Rev   Zj {Zjevení Janovo}     

\BookException Ž   {\def\amark{Z}}
\BookException Př  {\def\amark{Pr}}
\BookException Pís {\def\amark{Pis}}
\BookException Ř   {\def\amark{R}}
\BookException Žd  {\def\amark{Zd}}

\def\nochapbooks{Abd Fm 2Jn 3Jn Ju}
 % Book titles and marks
%
%\def\txsfile {sources/Cze\tmark-\amark.txs} % Location of .txs files
%\def\fmtfile {formats/fmt-Cze\tmark-\amark.tex} % Location of fmts
%\def\notesfile {notes/notes-\amark.tex} % Location of notes
%\def\introfile {others/intro-\amark.tex} % Location of introductions
%\def\articlefile {others/articles-\amark.tex} % Location of articles%
%
%\def\printedbooks {%
% Gn Ex Lv Nu Dt Joz Sd Rt 1Sa 2Sa 1Kr 2Kr 1Pa 2Pa Ezd Neh
% Est Jb Ž Př Kaz Pís Iz Jr Pl Ez Da Oz Jl Am Abd Jon Mi
% Na Abk Sf Ag Za Mal 
% Mt Mk Lk Jn Sk Ř 1K 2K Ga Ef Fp Ko 1Te 2Te 1Tm 2Tm 
% Tt Fm Žd Jk 1Pt 2Pt 1Jn 2Jn 3Jn Ju Zj
%}%
%
%\def\printedbooks{Da} 
%\processbooks % Generates books declared in \printedbooks
%\bye
%\endtt%
%
%\secc[sample-notes] Example note file
%
%\begtt
%\Note 1:0 {} 
%\switch {BKR}{Taking Daniel and his companions to the royal court, 
% 8. He was faithful to God, 
% 17. being endowed with special gifts from God, he was made precious to them in the same court. }% 
% {PSP}{}% 
% {CSP}{Daniel separated}% 
% {CEP}{God protects the faithful Judean Jinochim in the court of \x/Nabuchodonosor 
% and will give them wisdom above others.}% 
% {B21}{}%  
% {SNC}{}%
%
%
%\putArticle 1:18 {Redemption (\<1:18-19>)} [1] () %1Pt
%
%\ww {to the dead} %BKR
% {to the dead} %PSP
% {dead} %CSP
% {dead} %CEP
% {dead} %B21
% {dead} %SNC
% {} {} {} {}
%\Note 4:6 {dead} Theological universalism (the doctrine that all will be saved) relies 
on a second chance after death. 
%This state of affairs is aptly illustrated by a teacher who would assign a homework assignment, 
% and yet not set a due date for submission. 
%If infinite time were indeed available for the decision to ask Christ for forgiveness, 
no one would have done it.
%
%\putBot 2:1 {Daniel's Vision} [danielovyvize] () {
% \Heros \cond \setfontsize{at 10pt}\rm
% \inspic{Nabuco-crop.pdf}
%puttext 5mm 112mm {\bf Four Kingdoms}
% \puttext 110mm 112mm {\bf Chronology of Empires}
% \puttext 13mm 107mm {Vision ch. 2}
% \puttext 45mm 107mm {Vision of ch. 7}
% \puttext 67mm 107mm {Vision ch. 8}
% \puttext 90mm 107mm {Identification}
% \puttext 139mm 104mm {600}
% \puttext 139mm 90.5mm {500}
% \puttext 139mm 76.5mm {400}
%puttext 139mm 62.5mm {300}
% \puttext 139mm 49mm {200}
%puttext 139mm 45mm {167 BC}
%puttext 139mm 41mm {Maccabees}
%puttext 137mm 37mm {and the Hasmoneans}
% \puttext 139mm 32mm {100}
%puttext 139mm 21mm {\bf 0}
%puttext 139mm 2mm {100}
% \puttext 139mm 11mm {70 A.D.}
% \puttext 139mm 7mm {Fall of Jerusalem}
%\bf  
% \puttext 144mm 108.5mm {626 BC}
% \puttext 144mm 95.5mm {539 BC}
%puttext 144mm 67mm {330 BC}
%puttext 144mm 30mm {63 BC}
%\puttext 15mm 91mm {\x/Golden head/}   
% \puttext 48mm 91mm {Lev}
% \puttext 90mm 91mm {\x/Babylon/ (<2:37-38>)}
%\puttext 15mm 66mm {\x/Breasts and Shoulders}}   
%\puttext 17mm 62mm {\x/silver}   
%puttext 46mm 66mm {\x/Bear}   
%puttext 70mm 66mm   
% \puttext 90mm 66mm {Médo-Persie (<8:20>)}   
%\puttext 15mm 46mm {\x/Belly and loins}}   
%\puttext 20mm 42mm {\x/copper/}   
% \puttext 46mm 44mm {\x/Pard/}   
%puttext 70mm 44mm   
% \puttext 90mm 44mm {\x/Greece/ (<8:21>)}   
%\puttext 15mm 30mm {\x/Hnats}}
%\puttext 15mm 26mm {\x/iron/}
%puttext 45mm 30mm
%puttext 46mm 26mm
%puttext 90mm 30mm {Rome} 
%\puttext 10mm 2mm
%\puttext 15mm -2mm
%}
%
%\putArticle 5:20 {Who was \x/Darius/the Mede?} [6] ()
%
%\putCite 1:8 {We do not achieve secondary things by declaring them primary. 
%Secondary things are achieved only if first things come first. 
%\quotedby{C. S. Lewis}}
%
%\putCite 3:30 {Courage is not only one of the virtues, but is the form of every virtue at the point of trial, 
...that is, at the point of the most extreme reality.
%\quotedby{C.S.Lewis}}
%
%\putCite 5:4 {Power corrupts, and absolute power corrupts absolutely. {\quotedby {Lord Acton}} 
%
%\putCite 6:11 {The life of a Christian is a life in enemy-occupied territory. {\quotedby{C.S.Lewis}}
%
%\endinput
%
%\ww {} %BKR
% {} %PSP
% {} %CSP
% {} %CEP
% {} %B21
% {} %SNC
%Note 1:1 {}     
%
%\endtt
%
%\secc[sample-articless] Example article file %
%
%\begtt
%\Cite A
%{In literature and art, no one who prides himself on originality will ever be original: 
%while you're just trying to tell the truth (and it doesn't matter how many times it's been told), 
nine times out of ten you'll become original without even noticing it. 
%\quotedby {C. S. Lewis}
%}
%%\swapCites
%
%\Article [1] %Purchase 1Pt
%
%Imagine a scene from an action thriller: Your 16-year-old daughter is kidnapped by terrorists and
%demanding a ransom. %
%
%You sell everything you have, go into debt to the point of death to raise the money you need,
and then, with the money in your suitcase, you arrive at the agreed-upon spot in the middle of the bridge. 
the designated end of the bridge as you were instructed. A van emerges from the darkness, stops in the middle of the bridge, 
your daughter gets out, you're relieved she's alive and hopefully okay, she picks up the suitcase and throws it 
into the car, then she turns towards you, slaps her left hand on her right bicep while holding up 
her right fist with the back of her hand towards you while screaming: “Thank you, dumbass!” 
%And jumps in the van and disappears.
%
%\insertCite A\left\
%This spitting in the face of the love that sacrificed everything for us is what we do every time we put sin before holiness (\<"v." 15-16>).         
%%\endtt
%%Article, quotation, etc.
%
%\secc[sample-fmt] Example file `fmt`
%
%\begtt %CEP Jr
%\fmtpoetry{23:28}{true.//// grain?",///}
%\fmtpoetry{23:29}{// fire,/// of the Lord,///}
%\fmtins{23:28}{pravdivě.}{\medskip}
%\fmtpre{23:30}{\medskip}
%
%\fmtfont{23:28}{pravdivě}{\em}
%\endtt
%
%\secc[sample-intro] Example file for Introduction
%
\sec[sum] Summary of basic commands and definitions

I would put everything the user might need to set up in %Sem. If this could fit on two pages, we might recommend printing it as a cheat-sheet.

\secc Typically in the main file

\begitems
* \`\tmark`: `\def\tmark{<t-short>}`: declaration of the t-short of a translation, e.g. BKR, PSP.
* \`\txsfile`: `\def\txsfile {<filename mask>}`: declaration of the location of `.txs`. 
  You can use `\amark` or `\bmark` in the filename mask. 
* \`\notesfile`: `\def\notesfile {<filename mask>}`: location declaration
   annotation files. 
* \`\introfile`: `\def\introfile {<filename mask>}`: location declaration
   of files with book introductions. 
* \`\articlefile`: `\def\articlefile {<filename mask>}`: declaration of the location of the
   of article files.
* \`\fmtfile`: `\def\fmtfile {<filename mask>}`: declaration of the location of
   file formatting data. 
* \`\input` `{<filename>}`: read the next file in the given location.
* \`\printedbooks`: `\def\printedbooks {<list of a-bookmarks>}`:
  list of books to be processed by `\processbooks`.
* \`\processbooks`: instruction to process the Bible.
\enditems
The above commands need to be defined in the main file, 
read `books.tex`, `vars.tex` and finally start processing with the command
\`\processboks`.

\secc Typically in the file `books.tex`

\begitems
* \`\BookTitle` `<a-tag> <b-tag> <title>`: declares for each book
  a-tag, b-tag and title. As each individual book is processed, the
  `\def\amark{<a-mark>}` and `\def\bmark{b-mark}`. The a-brand is used in
  references in the text of the Bible, and the b-mark may be used in file names.
* \`\BookException` `<a-mark> {<code>}`: before processing a book, the
  <code> is executed.
* \`\BookPre` `<a-tag> {<code>}`: after printing the introduction before the main text
  the declared book, <code> is executed.
* \`\BookPost` `<a-tag> {<code>}`: after printing the main text
  of the declared book, <code> is executed.
* \`\nochapbooks`: `\def\nochapbooks{<list>}`: list the a-brands of the books that
  are not divided into chapters.
\enditems

\secc Typically in the file `vars.tex`

This file declares variant phrases for different versions of the Bible translation.

\begitems
* \`\variants` `<number>` `<list of t-short translation variants>`: declares the number of
  translation variants and a list of \hbox{t-strings}. The number must match the number of
  t-shortcuts as well as the number of parameters of the \`\vdef` and \`\ww` commands.
* \`\vdef` `<phrase variants>`; declares the variants of the phrase, the first of which is
  reference. The phrase variants are given by the literal `{<this>}`.
\enditems

\secc Typically in the notes file

\begitems
* \`\Note` `<verse number> {<commented phrase>} <note text terminated by a blank line>`:
  declares a note.
* \`\ww` `<variants of phrase>`: must precede \`\Note`. V
  each translation variant, the following note will be bound to the phrase
  of a particular phrase according to the currently defined t-tag.
* \`\x``/<phrase>/`: prints the phrase according to the current t-tag and according to
  \`\vdef` declaration. The <phrase> parameter is the reference phrase and is printed
  this parameter if no t-tag is defined or if the first
  t-character from the list of t-characters defined by \`\variants`.
* \`\putBot` `<verse number> {<title>} [<slash>] (<commands>) {<code>}`:
  sets up an image/diagram at the bottom of the page containing the <verse number>.
\enditems

\secc Typically in a book introduction file

... \TODO, add

\secc Typically in an article file

... \TODO, complete

\secc Typically in a file with formatting data

... \TODO, complete

\secc When creating maps

... \TODO, complete

\secc When creating timelines

... \TODO, complete




\sec Index

\def\_sortinglang{en}
\begmulti 3
\typosize[9/]
\makeindex
\endmulti

\bye




% what's below should be in the same Overleaf project with the corresponding 00-README in the individual directories.

\secc[summa-main] Main file `main.tex`

\begtt
\load[opbible] % macros OpBible
\cslang

%\checksyntax notes-Da intro-Da articles-Da fmt-BKR-Da {} % Error tracer; when active, \processbooks is disabled (does nothing)
%\normalchapnumbers % smaller chapter number (just over first 2 lines); no big quote marks next to the quotations
%\mergednotes % more than one note to a single verse make one paragraph but at the expanse of ignoring the search phrases
%\let\notecolor=\relax % disables \Red search phrases 
\def\shadowparameter{.1}%text background transparency; {1}=solid white
%\def\townparams{
% \hhkern=.8pt % radius of the sphere
% \lwidth=.5pt % contour line thickness
% \fcolor=\Red % ring color
% \lcolor=\Black % contour line colour
%} 


%\input hebrew % Hebrew phrases are declared here
%\input greek % Greek - "-

% Variants of translation:
\def\tmark {BKR} % Bible Royal
%\def\tmark {PSP} % Pavlik's study translation
%\def\tmark {CSP} % Czech study translation
%\def\tmark {CEP} % Czech Ecumenical Translation
%\def\tmark {B21} % Bible for the 21st century
%\def\tmark {SNC} % Word for the Journey

\input {Cze-vars.tex} % Variants of translation
\BookTitle Gen  Gn {První Mojžíšova (Genesis)}
\BookTitle Exod Ex {Druhá Mojžíšova (Exodus)}
\BookTitle Lev  Lv {Třetí Mojžíšova (Leviticus)}
\BookTitle Num  Nu {Čtvrtá Možíšova (Numeri)}
\BookTitle Deut Dt {Pátá Mojžíšova (Deuteronomium)}
\BookTitle Josh Joz {Jozue}
\BookTitle Judg Sd {Soudců}
\BookTitle Ruth Rt {Rút}
\BookTitle 1Sam 1S {První Samuelova}
\BookTitle 2Sam 2S {Druhá Samuelova}
\BookTitle 1Kgs 1Kr {První Královská}
\BookTitle 2Kgs 2Kr {Druhá Královská}
\BookTitle 1Chr 1Pa {První Paralipomenon (1. Letopisů)}
\BookTitle 2Chr 2Pa {Druhá Paralipomenon (2. Letopisů)}
\BookTitle Ezra Ezd {Ezdráš}
\BookTitle Neh  Neh {Nehemjáš}
\BookTitle Esth Est {Ester}
\BookTitle Job  Jb {Jób}
\BookTitle Ps   Ž {Žalmy}
\BookTitle Prov Př {Přísloví}
\BookTitle Eccl Kaz {Kazatel}
\BookTitle Song Pís {Píseň písní}
\BookTitle Isa  Iz {Izajáš}
\BookTitle Jer  Jr {Jeremjáš}
\BookTitle Lam  Pl {Pláč}
\BookTitle Ezek Ez {Ezechiel}
\BookTitle Dan  Da {Daniel}
\BookTitle Hos  Oz {Ozeáš}
\BookTitle Joel Jl {Jóel}
\BookTitle Amos Am {Ámos}
\BookTitle Obad Abd {Abdijáš}
\BookTitle Jonah Jon {Jonáš}
\BookTitle Mic  Mi {Micheáš}
\BookTitle Nah  Na {Nahum}
\BookTitle Hab  Abk {Abakuk}
\BookTitle Zeph Sf {Sofonjáš}
\BookTitle Hag  Ag {Ageus}
\BookTitle Zech Za {Zacharjáš}
\BookTitle Mal  Mal {Malachiáš}
\BookTitle Matt Mt {Matouš}
\BookTitle Mark Mk {Marek}
\BookTitle Luke Lk {Lukáš}
\BookTitle John Jn {Jan}
\BookTitle Acts Sk {Skutky apoštolské}
\BookTitle Rom  Ř {Římanům}
\BookTitle 1Cor 1Kor {První list Korintským}
\BookTitle 2Cor 2Kor {Druhý list Korintským}
\BookTitle Gal  Ga {Galatským}
\BookTitle Eph  Ef {Efezským}
\BookTitle Phil Fp {Filipským}
\BookTitle Col  Kol {Koloským}
\BookTitle 1Thess 1Te {První list Tesalonickým}
\BookTitle 2Thess 2Te {Druhý list Tesalonickým}
\BookTitle 1Tim 1Tm {První list Timoteovi}
\BookTitle 2Tim 2Tm {Druhý list Timoteovi}
\BookTitle Titus Tt {Titovi}
\BookTitle Phlm  Fm {Filemonovi}
\BookTitle Heb   Žd {Židům}
\BookTitle Jas   Jk {List Jakubův}
\BookTitle 1Pet  1Pt {První list Petrův}
\BookTitle 2Pet  2Pt {Druhý list Petrův}
\BookTitle 1John 1Jn {První list Janův}
\BookTitle 2John 2Jn {Druhý list Janův}
\BookTitle 3John 3Jn {Třetí list Janův}
\BookTitle Jude  Ju {List Judův}
\BookTitle Rev   Zj {Zjevení Janovo}     

\BookException Ž   {\def\amark{Z}}
\BookException Př  {\def\amark{Pr}}
\BookException Pís {\def\amark{Pis}}
\BookException Ř   {\def\amark{R}}
\BookException Žd  {\def\amark{Zd}}

\def\nochapbooks{Abd Fm 2Jn 3Jn Ju}
 % Book titles and marks

\def\txsfile {sources/Cze\tmark-\bmark.txs} % Location of .txs files
\def\fmtfile {formats/fmt-Cze\tmark-\amark.tex} % Location of fmts
\def\notesfile {notes/notes-\amark.tex} % Location of notes
\def\introfile {others/intro-\amark.tex} % Location of introductions
\def\articlefile {others/articles-\amark.tex} % Location of articles

\def\printedbooks {%
   Gn Ex Lv Nu Dt Joz Sd Rt 1Sa 2Sa 1Kr 2Kr 1Pa 2Pa Ezd Neh
   Est Jb Ž Př Kaz Pís Iz Jr Pl Ez Da Oz Jl Am Abd Jon Mi
   Na Abk Sf Ag Za Mal 
   Mt Mk Lk Jn Sk Ř 1K 2K Ga Ef Fp Ko 1Te 2Te 1Tm 2Tm 
   Tt Fm Žd Jk 1Pt 2Pt 1Jn 2Jn 3Jn Ju Zj
}

\def\printedbooks{Da} 
\processbooks % Generates books declared in \printedbooks
\bye
\endtt

 

\secc[summa-intro] File `intro-*.tex`

\secc[summa-notes] File `notes-*.tex`

\begtt
%\ww {}%searched phrase of each translation
% ={}printed phrase % BKR
% {}={} % PSP
% {}={} % CSP
% {}={} % CEP
% {}={} % B21
% {}={} % SNC
\Note 1:1 {} %phrase common to all translations, e.g. wisdom, otherwise empty group{}
            %={}%a phrase not in the Bible but printed, e.g., Vision of the four \x/selem/
        Text notes.
\endtt




\secc[summa-article] File `article-*.tex`

\secc[summa-quote] Command `\putCite`

\secc[summa-maps] Map description

\secc[summa-fmt] Format files `fmt-*-*-*.tex`

\secc[summa-vars] Translation variants file `Cze-vars.tex`






\vfill\eject

\sec Index

\def\_sortinglang{en}
\begmulti 3
\typosize[9/]
\makeindex
\endmulti

\bye